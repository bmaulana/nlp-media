\documentclass{report}
\usepackage{setspace}
%\usepackage{subfigure}

\pagestyle{plain}
\usepackage{amssymb,graphicx,color}
\usepackage{amsfonts}
\usepackage{latexsym}
\usepackage{a4wide}
\usepackage{amsmath}

\usepackage[super]{nth}  % To format ordinal numbers e.g. 1st, 2nd, 3rd, 4th, etc.
\usepackage{enumitem}  % Remove spaces before an enumerate
\usepackage[style=ieee, sorting=none]{biblatex}  % BibTeX  
\usepackage{url}  % urls
\usepackage{listings}  % code snippets
\usepackage{longtable}  % tables
\usepackage{tabu}  % tables
\usepackage{textcomp}  % symbols

\newtheorem{theorem}{THEOREM}
\newtheorem{lemma}[theorem]{LEMMA}
\newtheorem{corollary}[theorem]{COROLLARY}
\newtheorem{proposition}[theorem]{PROPOSITION}
\newtheorem{remark}[theorem]{REMARK}
\newtheorem{definition}[theorem]{DEFINITION}
\newtheorem{fact}[theorem]{FACT}

\newtheorem{problem}[theorem]{PROBLEM}
\newtheorem{exercise}[theorem]{EXERCISE}
\def \set#1{\{#1\} }

\newenvironment{proof}{
PROOF:
\begin{quotation}}{
$\Box$ \end{quotation}}



\newcommand{\nats}{\mbox{\( \mathbb N \)}}
\newcommand{\rat}{\mbox{\(\mathbb Q\)}}
\newcommand{\rats}{\mbox{\(\mathbb Q\)}}
\newcommand{\reals}{\mbox{\(\mathbb R\)}}
\newcommand{\ints}{\mbox{\(\mathbb Z\)}}

\setcounter{secnumdepth}{3}  % Add numbering to \subsubsection
\setcounter{tocdepth}{3}  % Include \subsubsection in ToC
\setlist{nosep}  % Remove spaces before an enumerate
\setlength{\LTpre}{1em}  % remove spaces before longtable
\setlength{\LTpost}{0em}  % remove spaces after longtable
\newcommand{\textapprox}{\raisebox{0.5ex}{\texttildelow}}  % '~' sign

\bibliography{bibliography.bib}  % Load bibliography

%%%%%%%%%%%%%%%%%%%%%%%%%%


\title{{\vspace{-14em} \includegraphics[scale=0.4]{ucl_logo.png}}\\
{{\Huge Using Natural Language Processing (NLP) to develop a pipeline to analyse media representation of people with disabilities in Web-based news articles}}\\
%{\large Collection and filtering of online news articles, comparison of open-source sentiment models, and applications of the technology
%}\\
}
\date{Submission date: \nth{30} April 2018}
\author{Bagus Maulana\thanks{
{\bf Disclaimer:}
This report is submitted as part requirement for the MEng Degree in Computer Science at UCL. It is
substantially the result of my own work except where explicitly indicated in the text.
\emph{The report may be freely copied and distributed provided the source is explicitly acknowledged}}
\\ \\
MEng Computer Science\\ \\
Catherine Holloway, Nicholas C. Firth}



\begin{document}
 
\onehalfspacing
\maketitle
\begin{abstract}

\textbf{Report Title:}  Using Natural Language Processing (NLP) to develop a pipeline to analyse media representation of people with disabilities in Web-based news articles
%: Collection and filtering of online news articles, comparison of open-source sentiment models, and applications of the technology

\textbf{Author’s Name:} Bagus Maulana

\textbf{Supervisor’s Name:} Catherine Holloway, Nicholas C. Firth

\textbf{Date and Year of Submission:} \nth{30} April 2018\\
%% Three paragraphs:

%% What the project is about, the principle aims and goals, and specific challenges.
Research into the representation of groups (e.g. women, youth) in the news media is common across different research fields, from social sciences to computer science.
A common approach is to manually analyse a small sample of a few hundred news articles and generalise an overall conclusion from that sample. 
Computational Natural Language Processing (NLP) could be used to process articles much faster and vastly increase the sample size, which could uncover further information from a text corpus, such as trends i.e. how the conclusion varies over independent variables such as time.

%% How you carried out the project and what work it involved. How you went about meeting the project goals.
This project explores the feasibility of developing a computational pipeline that performs data collection from online news sources, filtering, parsing (feature extraction), sentiment scoring, statistical analysis, and data visualisation.
This pipeline is then used in an experiment to collect articles from three major British online news publishers and show trends regarding how the terms used and sentiment in news articles varies over time and between different publishers when reporting news related to disability or people with disabilities.

%% The results and achievements of the project. What the outcome is.
Results indicated that for some metrics, such as moving average of sentiment score, and on certain keyword categories, minor trends over time and between different publishers are apparent, although inconsistent. 
While the approach showed promise in performing quantitative analysis upon large bodies of literature, and specifically in the media representation domain, it is highly recommended that future approaches to this media analysis problem improves upon this work by training a domain-specific filter and sentiment scorer with labelled data to improve the accuracy, and thus consistency, of sentiment scoring.

%% Half a page’s length.}
\end{abstract}
\tableofcontents
\setcounter{page}{1}


\chapter{Introduction} \label{Introduction} % 2-4 pages

%% Outline the problem you are working on, why it is interesting and what the challenges are.

% Talk about NLP
Natural Language Processing (NLP) encompasses a wide range of computational techniques for machine understanding of human (natural) language that are often used alongside each other.
The review article \cite{cambria2014jumping} defined Natural Language Processing as ``a theory-motivated range of computational techniques for the automatic analysis and representation of human language.''
The techniques that fall under the NLP umbrella include extracting term frequency distributions, text processing (e.g. tokenisation, stemming), part-of-speech tagging, text classification, information extraction (e.g. entity recognition), sentence structure parsing (parse tree), and sentiment analysis (or opinion mining) \cite{Nltk, liu2012sentiment}.
The computational models used in NLP range from simple rule-based models (e.g. counting words, or term frequency) to statistical machine learning and neural network models \cite{awesome-sentiment-analysis}. 
% Applications of NLP are used in everyday technologies, for example, information retrieval for search engines such as Google and Bing, and categorisation and topic modelling for recommendation engines used to suggest `similar' articles. % todo cite or remove

An advantage of NLP is that machines could process vast bodies of human-created literature (books, articles, posts, e-mails, messages, etc.) much faster than humans can, processing thousands of text documents per second. 
This allowed for high-level quantitative analyses of thousands or millions of text documents from a vast corpus to be feasible, which could uncover information previously inaccessible from manually reading only a small sample of documents and generalising from the sample.
This level of quantitative analysis could uncover trends and patterns from a text corpus, to answer questions such as ``How does the popularity of the term `mentally ill' increase or decrease year-on-year in British news media?'' 

% Other work in using NLP for media analysis
Applying computational NLP to perform meta-analyses over large text corpora has interesting potential applications in improving our understanding of the human world, such as analysing cultural trends quantitatively \cite{lansdall2017content}.
One study assembled a vast corpus of regional newspapers in the United Kingdom spanning 150 years to detect long-term patterns of cultural change, such as the increase of female representation in the news, or the popularity of trains and horses for transportation \cite{lansdall2017content}. 
This was achieved by analysing trends for $n$-gram frequency (a count of words or phrases in a text document) and named entities (known people, organisations, locations, etc.) in text.

More specifically in the domain of media representation of particular groups of people (e.g. women, youth), several researchers have attempted to use features extracted using NLP to perform computational analytics of text, mainly from social media.
For example, a tool to classify racist and sexist posts in social media was developed by using NLP to extract $n$-grams and part-of-speech tags (labels of words corresponding to its definition and context, such as `noun' or `verb') from text posts \cite{waseem2016you}.
However, there is still a research gap in this area, especially for applying NLP for news articles, in the context of specific groups such as people with disabilities.
% Just give a short summary, expand this later in Chapter 2

% Why is it interesting & what the challenges are
The representation of groups such as people with disabilities in media has been a popular research theme in social science.
For example, a 2002 study analysed a sample of 600 print articles relating to mental illnesses in New Zealand to measure the proportions of positive and negative depictions and predominant themes (e.g. criminality, educational accomplishments) \cite{coverdale2002depictions}.
There were attempts to discover trends, such as a study conducted in 1998 \cite{gold1999media} and replicated in 2008 \cite{devotta2013representations} which assessed change in representations of disability and people with disability in Canadian news media.
However, the study provided only two data points (1998 and 2008) with relatively small sample sizes of 196 news articles in 1998 and 166 news articles in 2008.

Applying computational NLP to this field would allow the possibility of discovering higher-level trends, by computationally analysing a much larger sample of articles, then identifying trends by creating subsets based on independent variables such as year of publication and publisher.
In this research, a sample of 305,185 news articles (48,967 after filtering off-topic articles) from British online news sources were used.
However, challenges remain as contemporary syntax-based NLP approaches tend to be more limited in scope and are prone to inconsistencies (false positives and negatives), where mitigating these inconsistencies is currently an open area of research. % inconsistencies due to type-I and type-II errors? (false positives and negatives)
% Just give a short summary, expand this later and give more papers in Chapter 2

%% List your aims and goals. An aim is something you intend to achieve (e.g., learn a new programming language and apply it in solving the problem), while a goal is something specific you expect to deliver (e.g., a working application with a particular set of features).
The aim of this project was to show the feasibility of utilising computational NLP approaches to perform a meta-analysis of literature available in the public online news media.
More specifically, to collect news articles relating to people with disabilities in British online media, and perform analyses using NLP at scale to identify trends such as term popularity and variations in positive/negative sentiment over variables such as date published and publisher.

This project's goals were to achieve the stated aim by developing a computational pipeline capable of performing analysis on online news media in full, from data collection to analysis and visualisation.
Given a list of topics related to disabilities, each topic consisting of key terms and query terms;
this pipeline accomplished the task of web crawling and scraping web sites to collect a dataset of news articles;
filtering relevant articles given key terms; 
extracting `relevant' sentences that referred to a key term from these articles;
performing sentiment analysis on these sentences;
and producing relevant visualisations and statistical analyses to show trends.
This pipeline is available open source on GitHub (\url{https://github.com/bmaulana/nlp-media}).

The main NLP techniques that were relevant for this project are:
Text processing, to parse text and other relevant information from web news articles, and `prepare' text for further analyses using tokenisation (splitting text into a list of tokens, or words) and stemming (reducing words to their word stem, e.g. talked \textrightarrow\space talk);
term ($n$-gram) frequency, to quantitatively count the occurrence of words and phrases (sequences of words) in an article;
sentiment analysis, to produce a `sentiment score' of news articles that correspond to its perceived positive/negative view.
These techniques were implemented by utilising open-source NLP implementations.

%% Give an overview of how you carried out the project (e.g., an iterative approach).
This project was carried out in a modular approach. The pipeline was developed as individual components: 
a web scraper and crawler for data collection given a list of queries; 
a filter to remove irrelevant articles given a list of key terms; 
a parser to extract term occurrences and relevant sentences from articles, given a list key terms; 
a sentiment scorer for sentences and articles; 
and a script to perform statistical analysis on the results and produce relevant plots. 
A main pipeline script connects these components together by calling them in sequential order, performing analyses for each topic and Web news source (Daily Mail, Daily Express, Guardian).
Each component's output is saved to a JSON \cite{rfc8259} file, and the next component reads the previous component's output file, which ensured that computation can be `resumed' without recomputing the previous component. % is this too much implementation details for intro? meant to show advantages of modular approach

%% A brief overview of the rest of the chapters in the report (a guide to the reader of the overall structure of the report).
The body of this report is subdivided into four chapters, followed by a conclusion and appendices.
Chapter \ref{Context} covers related work on the domain of news media analysis (especially in the context of people with disabilities), a background of NLP research, and information regarding the relevant technologies researched for this project.
Chapter \ref{Requirements and Analysis} defines a structured list of requirements, goals, and expectations for this project.
Chapter \ref{Design and Implementation} documents the design and implementation of the computational pipeline and its components that were used to carry out the data analytics experiment and achieve the stated goals.
Chapter \ref{Results Evaluation} discusses the experiment's results in diagrams, graphs, and tables.
Finally, the conclusion, Chapter \ref{Conclusions}, evaluates the project, summarises key achievements and takeaways, provides recommendation on how this work could be expanded upon, and sets guidelines for further work in this field.
Furthermore, the bibliography lists sources that were used as references in this project, and the Appendix section contains the raw results of the experiment, including graphs and data not in Chapter \ref{Results Evaluation}.

% Brief results & conclusions
The performance of several open-source sentiment analysis implementations were compared to judge their suitability for the sentiment scoring task, given the domain of sentences from news articles related to disabilities or people with disabilities.
Results indicated that a relatively simpler rule-based model, VADER \cite{VADER}, outperformed more complex supervised machine learning or neural network models, which had been trained on other domains, such as tweets and IMDb/Amazon reviews, and proved to be less generalisable for this domain.

This project proved the feasibility of utilising NLP-based technologies to derive trends from a large corpora of online news articles relating to disabilities or people with disabilities.
The results showed that, for example, the perceived sentiment of Guardian articles are, on average, significantly higher than Daily Express and Daily Mail articles for the topic `disabled'.
It also showed an decreasing trend over time on the use of `invalid' and `handicap(ped)', and an increasing trend for `accessibl(e)', within Guardian articles.

These results could have a substantial impact on how research should be conducted for similar studies on the media's representation of groups of people, as it showed that NLP could be used to analyse a much larger sample of text documents than traditional approaches and derive meaningful trends.
However, challenges remained due to the sentiment scorer's inaccuracy.
The accuracy and consistency of results could be further improved in future work by developing a domain-specific supervised model for filtering and sentiment scoring, trained on an adequately-sized labelled dataset of news articles within the domain.   

%% This chapter is relatively short (2-4 pages) and must leave the reader very clear on what the project is about and what your goals are


\chapter{Context} \label{Context} % 8-10 pages

%% This chapter should cover background information, related work, research done, and tools or software selected for use in the project.

%% Provide necessary context and background information to describe how your project relates to what is already known or available.

%% A description of the research carried out to learn out about the nature of the problem(s) being investigated and potential solutions. The form of the research will vary widely depending on the kind of project. For example, it might involve searching through research publications and online resources, or might involve an exploration of design possibilities for a user interface or program structure.

%% The sources of information you are drawing on (papers, books, websites, etc.) should all be cited or referenced clearly. In addition, state how each source relates to your work and avoid the temptation to pad out the chapter by including sources that you didn’t make use of during the project.

%% If relevant, a survey of similar solutions, programs or applications to yours, and how yours is differentiated.

%% Introduce the software, programming languages, library code, frameworks and other tools that you are using. Discuss choices and make clear which you made use of and why.

%% You should not include well known things (e.g., HTML or Java) or try to give tutorials on how to use a tool or code library (use references to books and websites for that information). Everything you include should be directly relevant to your work and the relationship made clear. This chapter is likely to be fairly substantial, perhaps 8-10 pages.

% -- How the project relates to what is already known & available, survey of similar solutions --
\section{Background} \label{Background}  % 3-4 pages

% Similar research being done on media representation of (disability/mental illness/etc.).
Analyses of news media has been a popular research method.
News media provides an overview of the prevailing society's conceptions or views regarding a theme or topic, which can be analysed to deduce quantitative information.
This approach has been commonly used to study representations of particular groups of people, such as women, youth, or people with disabilities, as the language used in the media reflected and shaped prevailing views, and had been shown to differ (with statistical significance) in different societies.
For example, it was shown that the Canadian press was more likely to name individuals with disability and use appropriate labelling than the Israeli press in 1998 \cite{gold1999media}.
Furthermore, there is evidence to suggest that news media sources contribute to shape and reinforce beliefs among the society, such as misconceptions and stigma \cite{wahl1992mass}.

There had been various studies related to the public awareness of disabilities.
% While not necessarily related to news media,
A review in 2011 \cite{scior2011public} found 75 articles and 68 studies that passed a selective inclusion criteria with regards to intellectual disabilities, published in English between 1990 and mid-2011. 
Their inclusion criteria omitted studies which were found to be irrelevant, duplicate, or not written in English; and only accepted articles which were published in full in peer-reviewed journals, and the study's subject had to be the general public of working age (instead of particular subgroups).
This showed interest within the research community to find new ways to understand and quantify the public's perception towards disabilities.
% The topics brought up include the public's knowledge, attitudes, and beliefs about intellectual disability; and varying for socio-demographic characteristics, cross-cultural comparisons, and the effects of interventions.

Analyses of news media were primarily carried out by obtaining a small sample of documents (news articles) from a text corpora, such as all news articles published in England for a certain period, and analysed them manually. There had been various such studies on news media within the domain of disability representations:
\begin{itemize}
	\item In a 2002 study \cite{coverdale2002depictions}, researchers analysed a sample of 600 print articles published in New Zealand relating to mental health or mental illness that were collected by a commercial clipping bureau.
		The articles were categorised into positive or negative depictions; and further into themes such as danger to others, criminality, vulnerability, etc.
		The study found that negative themes predominate about 3 to 1, with 27\% being positive.
		Given the study's scope, this conclusion cannot be generalised to learn trends of how the conclusion varies given certain variables such as time or location.
	\item A similar study in 2005 \cite{jones2009representations} analysed 1,515 articles relating to autism in Australian news media.
		All articles were read by two research assistants to ensure they are on-topic and then coded as either `negative' or `positive' in overall focus, and then coded into themes (e.g. funding, education, etc.).
		% Key findings include a relatively limited amount of helpful information, and a 'dual stereotype' of people with autism labelled as either dangerous and uncontrollable, or unloved and poorly treated. % is this relevant?
		% Relative to other similar studies, this study appears to be focused more on qualitative discussion compared to quantitative results. % is this relevant?
	\item A study conducted in 1998 \cite{gold1999media} and replicated in 2008 \cite{devotta2013representations} assessed change in representations of disability and people with disabilities in the Canadian news media.
		This study sampled 196 news articles in 1998 and 166 news articles in 2008.
		It found an increase in the usage of `person-first' terminology (e.g. person with disabilities) and a decrease in `disabling language' (e.g. disabled person). % is this relevant?
		This was an attempt to identify trends with regards to media representation of disability, however it only provided two data points (1998 and 2008) with relatively small sample size.
\end{itemize}

% Why using computational NLP will allow the discovery of more (high-level) information: trends, patterns, etc
Data collection and processing of text using computational techniques based on NLP are much more feasible in scale, cost, and time, relative to manual collection and reading of text. 
An automated script could be used to collect thousands of news articles published on the Internet per hour, a vast improvement over contracting a commercial clipping bureau to provide 600 articles as done in past studies \cite{coverdale2002depictions}.
By applying NLP-based computational techniques in analysing text, it would be possible to develop a pipeline that could analyse and extract quantitative information from news articles' text at a much faster pace than manual reading, enabling the analyses of a much larger scale of documents.

While a sample of few hundred documents was usually enough to provide statistically significant conclusions, by providing an analysis of the full corpora or a much larger sample, it would be possible to uncover additional information from the dataset.
Higher-level trends, such as how a conclusion varies by year, location, and publisher, could be discovered from a quantitative analysis of the larger dataset; by `splitting' the result set into smaller subsets based on independent variables such as year, location, and publisher, then comparing these subsets based on dependent variables such as term frequency.
Furthermore, computational pipelines for data collection and scoring news articles are more objective and reproducible than manual methods, which may vary due to each researcher's individual biases.

% Similar research on applying NLP for news media, social media, etc. in other domains - techniques & features they used, solutions they came up with, etc. - what could be applicable for this domain (sentiment relative to a certain topic)
There had been numerous attempts to take advantage of the NLP-based computational approach to conduct a more complete analysis of textual corpora. 
A team of researchers assembled a corpus of 35.9 million news articles from 120 publishers in the United Kingdom between 1800 and 1950, which represented 14\% of all news articles published in the United Kingdom over that period \cite{lansdall2017content}.
With a NLP-based computational approach, the researchers were able to extract quantitative time-series information, represented as $n$-grams and named entities, from the vast dataset, which allowed the researchers to discover macroscopic cultural trends. 
They analysed and compared $n$-gram trends across various topics and identified trends that reflect cultural shifts, such as `train' overtaking `horse' in popularity around 1900, or `labour party' overtaking `conservative party' and `liberal party' in news coverage from the 1920s.
Additionally, they used entity recognition to extract named entities, and identified trends based on known information about these entities, such as the proportion of female and male entities over time.
They also considered the geographical location of the publication, to identify how usage trends of $n$-grams such as `british' and `english' differ based on location.

The British news media study were inspired by prior discussions and studies on the potential of exploiting large text corpora to detect macroscopic, long-term cultural changes. 
A seminal study in 2011 \cite{michel2011quantitative} started the field of `culturomics', to perform large-scale quantitative analyses on text corpora.
In the seminal study, a corpus of 5 million digitised English language books published over 200 years, or about 4\% of all books ever published, were analysed to extract how often $n$-grams were used over time.
This data is available on \url{http://www.culturomics.org/}.
This information was then used to analyse trends in the English language: the size of the English lexicon, regularisation of English verbs from irregular to regular (`-ed') suffixes, or how quickly mentions of years (e.g. `1950') decline in use.
Influenced by the seminal study, numerous other studies adopted similar approaches to analyse large datasets: 
\begin{itemize}
	\item an analysis of 1.7 million Victorian-era books \cite{gibbs2011conversation}
	\item an analysis of 17,094 US Billboard Hot 100 songs between 1960 and 2010 \cite{mauch2015evolution}
	\item an analysis of a 3.9 million news articles sampled from the Summary of World Broadcasts (SWB) collection \cite{leetaru2011culturomics}
	\item an analysis of 2.5 million English-language news articles from 498 online news outlets from 99 countries \cite{flaounas2013research}
\end{itemize}
This approach had also been criticised for ignoring semantics and context.
For example, critics noted that ``thirteen hundred words of gibberish and the Declaration of Independence are digitally equivalent'' \cite{gooding2013mass}, issues with OCR quality and duplicate editions \cite{gooding2013mass}, or that the selection of digitised books were biased \cite{schwartz2011culturomics}.

There had been some progress in applying similar NLP-based approaches specifically in the domain of how specific groups of people are represented in the media:
\begin{itemize}
	\item An attempt to classify racist and sexist posts in social media, which used NLP to extract features (such as $n$-grams and part-of-speech tags) from social media posts, then used to annotate 6,909 tweets \cite{waseem2016you}.
	\item A study explored potential linguistic markers of schizophrenia in social media; using a dataset of 174 users with self-reported schizophrenia and up to 3,200 tweets per user, and a similarly-sized dataset of `control' (non-schizophrenic) users.
		The researchers used NLP to extract features based on lexicon-based approaches (i.e. a list of mental health related keywords), latent dirichlet allocation (LDA), Brown clustering, character $n$-grams, and perplexity from tweets, which were fed to a support vector machine (SVM) classifier \cite{mitchell2015quantifying}.
\end{itemize}
Despite these studies, there is still a visible research gap on applying computational NLP approaches with regards to the representations of specific groups, such as people with disabilities (or a specific disability).
This project focused on studying the representations of people with disabilities from online news articles.

% Research on relevant libraries, frameworks, etc.:
To date, advances in NLP research had made the `culturomics' approach much more feasible, efficient, and effective, even under limited time and hardware constraints.
A growing number of free and open-source tools for computational NLP and statistical analysis had been developed by the research community; including general NLP tools such as nltk \cite{Nltk}, SpaCy \cite{SpaCy}, and StanfordNLP \cite{StanfordNLP}, statistical analysis tools such as numpy \cite{Numpy}, scipy \cite{Scipy}, and scikit-learn \cite{Scikit-learn}, and data visualisation tools such as matplotlib \cite{Matplotlib}.
Section \ref{Technical Context} contains a further listing and discussion of these tools, alongside the specific libraries and NLP techniques used for this research project.

% -- Research carried out & sources (papers, books, websites, etc.) --
\section{Research Methodology and Sources} \label{Research Methodology and Sources}  % 1 page

% How background research was carried out
Background research was carried out by investigating papers from public sources such as Google Scholar.
Research articles were gathered from a list of important topics and query terms related to NLP, specific NLP techniques, disability, and news media: for example, `natural language processing review', `news media' AND `disability', `natural language processing' AND (`news media' OR `cultural trends'), `natural language processing' AND `disability', and `sentiment analysis'.
Highly-cited research articles were prioritised as examples, as they were deemed to be more `important' papers or studies within its topic.
Additionally, the author looked for highly-cited `key' papers and review articles within each specific topic, and consulted its list of citations (older papers) and research articles that cites the review article (newer papers) to expand the list of relevant research papers further.

% How technical research was carried out: for each component, determine neccesary or potentially useful techniques. Look up online for papers defining it, web resources, frameworks/libraries. Research done alongside development of code (do research as I plan/develop, which in turn may change how I'm doing things)
Technical research, on the other hand, was carried out as necessary.
After the requirements and components for the pipeline had been decided, research was carried out to find relevant techniques, formulae, algorithms, tools, libraries/packages, and existing implementations that would be useful to implement each necessary component.
This research was carried out in an iterative approach alongside software development. 
Initial planning and research would provide an initial implementation plan, the plan's implementation would uncover the feasibility of these approaches and possible alternatives/refinements to be researched, further research may reveal new options/refinements to be implemented, and so on.
Research or work done for a component may also reveal possible improvements and/or alternatives for another component, which may require further research and implementation.  
Again, more popular tools and libraries were prioritised; although several approaches and implementations were considered for most components and sub-tasks, to be compared for suitability, runtime, results, etc.

% Sources: Google Scholar, papers cited by other papers, GitHub pages, Python repositories (Pip, Conda), Python and library documentation, etc. + 'state how each source relates to your work'.
The sources that were used for this literature review are:
\begin{itemize}
	\item Google Scholar, often used to find research articles to act as `entry points' towards a research topic, to find other studies similar to a research article, or to retrieve citation information with regards to research paper, book, or popular Python package.
	\item GitHub topics, used to find repositories that are relevant to help implement a specific NLP technique or component, find similar/alternative implementations, and was also used as a benchmark for the popularity and range of solutions for a topic in a given programming language.
	\item Several GitHub pages also curate a list of repositories for a specific topic \cite{awesome-sentiment-analysis, awesome-nlp, awesome-machine-learning}.
	\item Python package repositories such as PyPi \cite{PyPi} and Anaconda Cloud \cite{Anaconda-Cloud}, which listed all available Python packages, provided general information regarding them, and a search function; useful to find relevant packages for a component/task and gather information about a Python package.
	\item Official web sites and documentation of Python packages, which listed and defined the capabilities of the package, used to explore a package's functionality and capacity to solve a specific task, and to understand the technologies and approaches used by the package's implementation. 
		On more popular packages, citation information regarding the package would often be available from its web site.
\end{itemize}

% -- Technical background, sources of information, software --% (probably the longest section in chapter 2)
\section{Technical Context} \label{Technical Context}  % 4-5 pages

% general NLP: define NLP again, classify & list techniques, then mention 'the techniques used in this project are ...'
As mentioned above, computational implementations of NLP techniques are utilised to extract features from collected news articles (text documents) at scale.
For this project, the requirements for the computational pipeline can be subdivided to five main components: data collection, filtering, sentence matching, sentiment scoring, and statistical analysis and data visualisation.
Among these components, NLP techniques are necessary for filtering, sentence matching, and sentiment scoring.
On the other hand, collection is performed using established, general-purpose web-scraping tools.
Similarly, statistical analysis and data visualisation is performed using general-purpose statistical tools and metrics.
This section will cover the technical tools researched and used for all listed components regardless.

NLP covers three main `curves' or areas:  syntax, semantics, and pragmatics (narratives, understanding). 
Syntax specifies the way symbols (words, terms, tokens, or $n$-grams) and groups of symbols are arranged and whether they are well-formed in an expression, whereas semantics specifies what these expressions mean, and pragmatics specifies contextual information \cite{cambria2014jumping}.
Contemporary approaches to NLP mainly focus on syntactic analysis, due to the relative ease of extracting syntactic features of text such as term frequency, word co-occurrence, and part-of-speech tags, compared to extracting logical expressions and networks necessary for semantic analysis.
However, syntactic analysis is much more limited as it often misses information such as the (semantic) context of a word, for example, the word `one' in ``there's no one there'' (referring to a person) vs ``we have only one car'' (referring to a quantity).
This paper will focus on mainly syntactic techniques and features, as these are more relevant to this domain of high-level topic matching and sentiment analysis that is feasible with current technology at this scale.

% Programming language: written in Python
Python was chosen as the main programming language used for this project.
The primary reason for this choice is the wide availability and range of existing tools for NLP, sentiment analysis, statistical analysis, data visualisation, web scraping and parsing, etc. in Python.
A study in 2016 showed that Python is the most popular language for machine learning and data science \cite{puget2016most}, which correlates to the amount of available tools developers have created for the language.
A GitHub search for the topic `nlp' as of 18 April 2018 reported 1,397 Python and 470 Jupyter (a Python-based interactive `notebook' technology) repositories with the tag `nlp', compared to the second most popular programming language being Java with only 251 repositories tagged with `nlp' \cite{GitHubNLP}.
Furthermore, Python is also an ideal language for quick experimentation due to relatively high-level and low verbosity of the code, such that it is relatively easier to make small changes on the fly.
The Anaconda distribution of Python \cite{Anaconda} is used for its suitability to set up and manage Python environments and packages for data science projects.

\subsection{Data Collection Methods} \label{tc-data-collection}
% Data collection
For data collection, general-purpose tools for sending HTTP requests (to `open' web addresses and store HTML web pages programmatically) and parsing HTML code (to parse article text and metadata from `raw' HTML code) are sufficient. 
The Requests library \cite{Requests} is a popular Python tool (with 400,000+ daily downloads) for sending HTTP/1.1 requests simply.
A HTTP GET request will retrieve the HTML code (and other information) associated with a given URL from a web server, similar to opening the page on a web browser, stored as a Python object by Requests.
Once the HTML code of a web page (given an article's URL) had been stored, the BeautifulSoup library \cite{BeautifulSoup} provides simple methods to navigate and search a parse tree (such as HTML code).
Given that web pages from the same source/publisher tend to follow a similar structure, BeautifulSoup can be used to parse article text and relevant metadata (e.g. headline, date of publication, outgoing links in a search page) by searching for specific tags and attributes within the HTML code.

\subsection{Filtering Methods} \label{tc-filtering}
% Term frequency & filtering: mention 'tf ranking' & pre-processing (stemming)
In this project, filtering of off-topic articles were achieved via ranking the term frequency of key terms independently for each document.
Term frequency (tf) is a simple and commonly-used metric in NLP, with various existing tools that can compute this metric for thousands of text documents within seconds.
To put simply, the frequency of a term (a token, or sequence of tokens) in a document is the number of times that the term occurs in the document.
The popular scikit-learn library \cite{Scikit-learn} provides a tool to measure term frequency of text documents, handling both tokenisation (converting a text document into a list of tokens, or terms/words) and counting word occurrence.
It also provides the option to ignore stop words (common words in English which do not add topical information, such as `the' or `a'), which can often bias results due to their relative high frequency.
Additionally, the nltk library \cite{Nltk} is used for `stemming': to reduce all words in the document to its word stem, such that e.g. `walk', `walks', `walked', and `walking' are equivalent.

In literature related to NLP, more complex approaches has been proposed and used for the task of text classification and filtering off-topic articles.
A conventional approach is by calculating term frequency --- inverse document frequency (tf-idf) \cite{robertson2004understanding, sparck1972statistical}, a metric that builds on term frequency by taking into account the relative importance of each word.
Inverse document frequency (idf) is calculated by counting the number of documents in a corpus where a term appears: if a term appears more frequently, it is deemed to be less important and assigned a lower score.
For example, common words such as `the' are assigned very low scores.
However, this approach was not suitable for this project, given the selective nature of the dataset, as only articles containing certain query terms are collected; thus the idf values of query terms were be flawed, as a query term exist in every document.

Another proposed approach is by using a supervised machine learning model to classify documents into pre-defined categories (the text classification problem).
Various approaches were proposed to solve text classification, including Support Vector Matrices (SVM), Na\"{i}ve Bayes (NB), and k-nearest neighbour (kNN) models \cite{khan2010review}.
However, this approach was not feasible for this project due to a lack of labelled data (i.e. a dataset of articles and category labels, or in this case `is this article relevant?' boolean labels).

Topic models, which compute the proportion of abstract `topics' in a document, has also been proposed.
Latent dirichlet allocation (LDA) \cite{blei2003latent} represents documents as random mixtures over latent topics, where each topic is characterized by a distribution over words.
However, as these topics are abstract and characterized generatively (i.e. each topic's distribution over words are generated by the model, rather than pre-defined), it is not very useful for the task of classifying whether a document matches pre-defined topics/keywords.
Additionally, LDA is significantly more computationally expensive than tf or tf-idf.

\subsection{Text Parsing and Sentiment Analysis Methods} \label{tc-sentiment}
% Rule-based matching to find sentences relevant to one or more keyword(s) or key phrase(s)
% match similar sentences if it contains a phrase that has the same lemma (root word) as the keyword or key phrase (in the same order), for example, 'Mental illnesses' will match 'mental illness'
% Cite SpaCy, 'used SpaCy to perform rule-based sentence matching'
Sentiment scoring of articles is sub-divided into two components: a component to find sentences relevant to a topic, or a list of key terms, in a text document (sentence matching); and another component that performs sentiment analysis on these sentences, and transforms sentences to a real-valued sentiment score. 
SpaCy \cite{SpaCy} is a popular tool for general natural-language processing tasks, using pre-trained convolutional neural network models for tasks such as tagging, parsing, and entity recognition, and is benchmarked to be the fastest and among the most accurate syntactic parser, able to parse 13,965 words per second in 2015 \cite{choi2015depends}.
Among the information SpaCy extracts from text are lemmas (root words) of terms (e.g. `mentally' \textrightarrow\space `mental') and features a rule-based matching engine (retrieve a list of sequences of tokens within a document that matches a given pattern, e.g. tokens with a specified lemma), both which are useful for the task of finding sentences relevant to a topic in a document.

% Sentiment analysis: cite some survey papers, cite GitHub page, cite each library I tried (this would probably be the longest section)
For sentiment scoring of sentences, a variety of open-source tools and pre-trained models dedicated to sentiment analysis were researched for the purpose of comparison.
Sentiment analysis, or opinion mining, is defined as "the field of study that analyzes people's opinions, sentiments, evaluations, attitudes, and emotions" from natural language \cite{liu2012sentiment}.
A GitHub search for the topic `sentiment-analysis' as of 18 April 2018 reported 450 Python and 222 Jupyter repositories with the tag `sentiment-analysis' \cite{GitHub-sentiment-analysis}.
Furthermore, a community-curated list of sentiment analysis methods and implementations exist \cite{awesome-sentiment-analysis}, which served as a useful starting point to explore open-source sentiment analysis implementations.

The particular sentiment analysis implementations that have been explored in this paper are:
\begin{itemize}
	\item VADER \cite{VADER} is a relatively simple parsimonious rule-based model that scores the sentiment of a given sentence, based on rules such as: the presence of `sentiment lexicons', a list of lexical features common to sentiment expression, such as `good' and `bad', including slang words, emoticons, and acronyms; negations (e.g. `not good'); and `emphasis', or increased sentiment intensity due to punctuation, capitalisation, and degree modifiers (e.g. `very').
	\item xiaohan2012's `twitter-sent-dnn' repository provides a trained a convolutional neural network model with dynamic k-max pooling (DCNN) for modelling (real-valued sentiment scores of) sentences.
		The sentence model properties considered are the word and $n$-gram order, and induced feature graph (generated by the DCNN).
		It was trained on a dataset of 1.6 million tweets with inferred labels based on emoticons.
		It is an implementation of \cite{kalchbrennerACL2014}.
	\item kevincobain2000's `sentiment\_classifier' repository \cite{kevincobain} provides a trained supervised machine learning model based on a Na\"{i}ve Bayes and Maximum Entropy Classifier to transform a sentence to positive and negative (real-valued) sentiment scores.
		It uses bigrams as features, implements Word Sense Disambiguation using wordnet \cite{banerjee2002adapted} to tranform bigrams to `senses', and considers word occurrence statistics from nltk's movie review corpus.
		Its training data is a mixture of nltk's movie review corpus, Twitter posts, and Amazon customer reviews data. 
	\item OpenAI's `generating-reviews-discovering-sentiment' repository provides a pre-trained single-layer multiplicative LSTM recurrent neural network model with 4096 units (a relatively simple model optimised for training/convergence time) to generate (real-valued) sentiment scores of input sentences.
		It was trained on a dataset of over 82 million Amazon product reviews from May 1996 to July 2014, substantially larger than previous work (and taking one month across four NVIDIA Pascal GPUs to train), and outperforms state-of-the-art models when tested on similar-domain corpora such as Rotten Tomatoes and IMDb reviews.
		Sentences are represented as a sequence of UTF-8 encoded bytes where for each byte, the model updates its hidden state and predicts a probability distribution over the next possible byte. 
		It is an implementation of \cite{OpenAI}.
	\item Stanford CoreNLP \cite{StanfordNLP} provides a set of linguistic analysis tools, including sentiment analysis, given input text, while running in a local web server.
		Its sentiment analysis tool uses a recursive neural network model, represents text as parse trees, and were trained on a Sentiment Treebank of fully-labelled parse trees for 215,154 unique phrases and 11,855 sentences from the Rotten Tomatoes movie review corpus. 
		Unlike other scorers in this list, it classifies sentences into five sentiment classes, from `very negative' to `very positive', instead of assigning a real-valued score. \cite{socher2013recursive}.
		Although Stanford CoreNLP was written in Java, several packages exist that allow a Stanford CoreNLP local server to be started and queried programmatically in Python \cite{stanfordcorenlp}.
	\item TextBlob \cite{textblob} is a general-purpose NLP library similar to nltk, SpaCy, or CoreNLP.
		It provides two sentiment analysis models: PatternAnalyzer, a rule-based classifier based on part-of-speech pattern matching, and NaiveBayesAnalyzer, a Na\"{i}ve Bayes classifier trained on a dataset of movie reviews (with no information on features used or dataset size). 
\end{itemize}
Several other repositories has also been explored, however deemed unsuitable for this project either due to requiring to be re-trained using labelled training data (which was unavailable for the domain of news articles), or the implementations are broken or infeasible.

\subsection{Statistical Analysis and Data Visualisation Methods} \label{tc-visualisation}
% Statistical analysis: matplotlib, papers (or at least a link to matplotlib's docs) for each statistical tool used (mean, standard deviation, scatterplot, moving average, 2d histogram, box and whiskers plot, violin plot, Mann-Whitney U test, etc.)
For statistical analysis and data visualisation, the conventionally used libraries in Python are numpy \cite{Numpy}, scipy \cite{Scipy}, scikit-learn \cite{Scikit-learn}, and matplotlib \cite{Matplotlib}. 
Numpy provides a powerful and efficient $n$-dimensional array object (often used as requirement for other libraries), and functions to perform mathematical operations over real values, vectors (1-dimensional arrays), and matrices (2-dimensional arrays) such as scalar/vector/matrix addition, multiplication, extracting columns of a matrix to a vector, and boolean filtering \cite{Numpy}.
Scipy is a library that extends numpy to provide additional domain-specific functions, providing tools such as sparse matrices and implementations of statistical equations such as estimating distributions \cite{Scipy}. 
Scikit-learn provides implementations of algorithms for data analysis, feature extraction, and machine learning, such as the CountVectoriser used to compute term frequencies \cite{Scikit-learn}.
Matplotlib is a 2D plotting library that produces visual graphs from lists/arrays \cite{Matplotlib}.
It provides the capability to generate various types of plots, such as scatterplots, line plots, histograms, and box-and-whisker plots; modify the plot parameters (such as colours, labels, and bounds), create a grid of axes and plot multiple graphs in the same axes, generate a legend or colorbar, among other features.

The types of plots and statistical analysis metrics that are deemed relevant for this analysis are:
\begin{itemize}
	\item Scatter plot, used to show the distribution of data within two variables (e.g. year published and sentiment score), and colour could be added to show a third variable (e.g. source/publisher of article).
		Available on Matplotlib.
	\item Histogram (and two-dimensional histogram), used to show the distribution of articles relative to variables such as publisher, year of publication, and sentiment score ranges. 
		Available on Matplotlib.
	\item Box-and-whiskers \cite{tukey1977exploratory} and violin plot \cite{hintze1998violin}, also used to show and compare the distribution of dependent variables (e.g. sentiment score) within different subsets of the data separated by independent variables (e.g. publisher).
		Available on Matplotlib.
	\item Line graph, used to show trends in a dependent variable (e.g. sentiment score) over an independent variable (e.g. year of publication).
		Available on Matplotlib.
	\item Mean and standard deviation, to quantify the distribution of articles within different subsets of the data separated by independent variables (e.g. publisher and year of publication) and provide a quantitative measure to compare different subsets.
		Available on SciPy.
	\item Mann-Whitney $U$ Test \cite{mann1947test}, a non-parametric statistical test that measures whether it is true that given a randomly-selected value from a distribution, and another randomly-selected value from another distribution, the first value is equally likely to be less than or greater than the second value (i.e. there are no statistically significant difference between the two distributions), to show if the difference between two subsets are statistically significant.
		Available on SciPy.
\end{itemize}

\chapter{Requirements and Analysis} \label{Requirements and Analysis}  % 5-6 pages

%% Give the detailed problem statement. This elaborates on what you may have included in the introduction chapter, and represents the starting point from which requirements were derived.
\section{Problem Statement} \label{Problem Statement}
% This project involves the development of a computational pipeline capable of performing data collection, natural language processing, statistical analysis, and data visualisation tasks

% Expand aims & goals
The primary aim of this project is to utilise available NLP-based technologies in order to perform a high-level meta-analysis of online news articles relating to people with disabilities available in the online British news media, with the goal of revealing trends by varying for independent variables such as source and year published.
The solution would need to collect and scrape online news articles from the Internet, filter only relevant articles, use available NLP-based tools to extract information from these articles, perform statistical analyses, and show visualisations of the resulting data to show trends.
Of particular interest, as a dependent variable, is a sentiment (or opinion/polarity) index of articles (``how positively does an article view disabilities or people with disabilities?''), and how it varies given independent variables such as source and year published.
Thus, the general solution defined by this aim would involve the completion of several sub-problems, primarily data collection, filtering, parsing, sentiment scoring, and statistical analysis and visualisation.

To achieve this aim, the project's goals are to develop a computational pipeline that implements all components required for the general solution.
Before the project could be started, the topics relevant to this project had to be defined.
Thus, a list of topics relating to the domain of people with disabilities would need to be compiled, each topic consisting of a list of keywords (or key phrases) and query terms related to a specific disability (or disabilities in general).
Given this list of topics, this pipeline has to implement these following tasks (goals):
\begin{itemize}
	\item Web crawling and scraping web pages, using public APIs where possible, to collect a dataset of news articles, given the list of query terms for each topic.
	\item Filtering relevant articles from the dataset, given the list of keywords for each topic.
	\item Extracting relevant sentences that refer to a keyword, from the dataset of filtered articles (feature extraction).
	\item Performing sentiment analysis (using open-source libraries and pre-trained models) on the dataset of relevant sentences.
	\item Producing relevant statistical analyses and data visualisation to show trends over independent variables such as source and year published.
\end{itemize}

As the primary advantage of computational NLP is in its speed, and thus analysed sample size, relative to human reading, the solution must be able to perform these computations quickly and at scale.
The total number of documents available, from the sources used in this experiment (section \ref{sources}) and given defined query terms (section \ref{topics}), is expected to number in the thousands to tens of thousands of documents per topic, or hundreds of thousands of documents in total across all topics.
Given this scale, the solution must be able to compute the full pipeline within a feasible timeframe (not more than a few days), given available consumer-grade hardware (Intel i7-6700HQ CPU @ 2.60GHz, NVIDIA GeForce GTX 1060 GPU) to show that the solution is feasible without specialised hardware.

%% A structured list of requirements.
\section{Requirements} \label{Requirements}

% Short paragraph about component based design
The solutions follows a component-based design, where each sub-problem must be implemented by a component that focuses only on the sub-problem.
These components must be linked together via a `pipeline' script that executes each component in order, and iterates through the list of topics and sources.
This is ideal such that changes could be made to a component without affecting (or needing to re-write) code in other components.
The list of required components are as follows: data collection, dataset filtering, rule-based sentence matching, sentiment scoring, statistical analysis and visualisation.

% Subsection for each component

\subsection{Data Collection} \label{req-data-collection}  % How to collect that data

% crawler finding articles from each source's built in search engines, look up multiple query terms
The data collection component must be able to find a list of news articles for each supported online source (Daily Mail, Daily Express, and Guardian), given a list of query terms related to each topic.
For each article, the only information required at this stage is a working URL pointing to an online resource containing the article text and relevant metadata.
Thus, the information that has to be provided by the component after this stage is a list of URLs pointing to relevant articles given a list of search queries.

% scraper for each source - what needs to be scraped from each article
After the list of URLs pointing to online news articles had been compiled, the component must be able to scrape these articles and extract the full article text and relevant metadata from each URL.
Aside from the full article text, the relevant metadata that needs to be extracted are the article's headline, URL, date of publication, and publication source.
The information is returned in the form of an array of JSON objects, with each JSON object containing the text and metadata of a single news article.
This information must then be saved locally to a file where it will be read by the next component. 

The file and directory structure of the output file must be consistent given source and topic, such that the next component can find it programmatically.
This requirement holds for all other components in this pipeline.

% scalability, able to save and resume progress, etc.
As data collection is expected to consume the longest time compared to other components, due to the necessity to submit a web request for each article, additional requirements regarding scalability are enforced.
The data collection component must run in reasonable time (i.e. less than a day for each topic), given available hardware and a scale of up to tens of thousands of articles per topic.
Additionally, it should be possible to resume progress on data collection, such that it is possible to re-run the program at a later date to add new articles without sending web requests for articles already in the collection.
It would also help in cases where the program is interrupted, Internet connection is lost, the machine is shut down, etc.
Thus, the component should be able to store already-parsed articles to an external file, and read from the external file upon starting to gather a list of already-parsed URLs, and avoid re-parsing existing URLs.

\subsection{Dataset Filtering} \label{req-filtering}

The next component handles dataset filtering.
Initially, this component must be able to load the dataset of parsed articles from the file saved by the data collection component.
Each file contains a subset of all parsed articles for a specified topic and a specified source.
For each article, it must decide whether it is relevant to the topic defined by the subset, given a list of key terms related to each topic.
This decision should be based by the article's full text and headline, and should take all key terms associated to the current topic into account.
It should also be able to show/print a sample of an arbitrary number of documents, which would be used to analyse and improve the accuracy of the filter.

The component must save all articles deemed on-topic (relevant to the topic) to a new output file, containing an array of JSON objects in the same format and with all the same information as in the data collection component's output file.
Articles deemed off-topic (not relevant to the topic) must not be saved to the output file.

\subsection{Feature Extraction and Rule-based Sentence Matching} \label{req-matching}

The next component parses the article text using open-source NLP tools to extract relevant information required by the sentiment scoring and statistical analysis components.
Given an input file which is the output file of the dataset filtering component, this component must be able to load the article text and metadata of all saved articles.
Then, information should be extracted by syntactically parsing each article's text and headline.
The sentiment scoring component would require all sentences relating to a keyword or key phrase (containing a keyword, key phrase, or an equivalent term) to be extracted from each article.
Additionally, the component should also extract other information as required by the sentiment scoring and statistical analysis components, including the total number of sentences and the number of relevant sentences in the document, and the term frequency of each keyword and key phrase.

After these data have been extracted from each article, it must save the information to a new output file, in the form of an array of JSON objects where each JSON object contains all the information about a single news article (with a key for each `feature' e.g. relevant sentences).
The metadata of each article (headline, date of publication, and source) should also be saved to the new output file, as it would be required as independent variables for the statistical analysis component to analyse trends in the dataset.

\subsection{Sentiment Scoring} \label{req-sentiment}

The sentiment scoring component computes a real-valued score for each relevant sentence extracted by the previous component, which must correspond to the perceived `sentiment' of the sentence towards a disability, disabilities, a person with a disability or disabilities, or people with disabilit(y/ies), referred by the keyword or key phrase, with sufficient accuracy.
Given an input file which is the output file of the dataset filtering component, this component must be able to load the relevant sentences and other data for all saved articles.

Two iterations of the sentiment scorer component must be developed.
The first iteration of the sentiment scorer component is used to perform a comparison between several open-source sentiment analysis implementations for this sentiment scoring task.
It is not used in the final pipeline.
The second iteration of the sentiment scorer component only computes one sentiment score for every sentence, using the best-performing sentiment scorer shown by the first iteration.
The second iteration is the one used in the final pipeline.

% Analysing scorers: 'slow' scorer that takes the first N articles only and scores them with ALL scorers from a list of open-source sentiment scorers, for the purpose of evaluating the performance (accuracy) of these scorers
\subsubsection{Comparison of Open Source Implementations} \label{req-sentiment-comparison}

The first iteration of the component must select a sample of sentences from the dataset, with a sample size arbitrarily defined by the user.
Also, it must implement all open-source sentiment analysis implementations (`sentiment scorer') that were listed in section \ref{tc-sentiment}.
All sentences in the sample must be analysed and given a score by each sentiment scorer, and the component should also allow the user to manually label these sentences as positive, neutral, or negative.
With this information, the component must be able to determine the accuracy of each sentiment scorer (i.e. show the proportion of true positives and true negatives, and show a confusion matrix), and store the total, per-sentence, and per-article runtime of each sentiment scorer.

This iteration of the component will not be used in the final pipeline, but only used as a tool to compare existing sentiment scorer implementations, and analyse their performance in the domain of sentences in news articles relating to disabilities or people with disabilities.

% Final product: one or two final sentiment scorers with only one sentiment score, optimised for performance (speed / runtime)
\subsubsection{Final Implementation} \label{req-sentiment-final}

The second iteration of the sentiment scorer component is used in the final pipeline.
This component must perform sentiment analysis for all relevant sentences in the dataset (instead of only a limited sample of sentences).
It must only compute one sentiment score using the best-performing sentiment scorer that runs in reasonable time, as shown by the first iteration of this component.
It should also compute the average sentiment score for each article, given information of the sentiment scores for all relevant sentences in the article.
Furthermore, the component must run in reasonable time (i.e. less than a day for each topic), given available hardware and a scale of up to tens of thousands of articles per topic.

After the sentiment scorer has scored all relevant sentences in the dataset, it must then save information about all articles, with score labels appended to each sentence, to a new output file.
The JSON object format of each article in the output file should be equivalent to the input file's format (i.e. no information from previous components are lost), with the exception of an additional `sentiment score' key-value pair within each sentence's JSON object.

\subsection{Statistical Analysis and Visualisation} \label{req-visualisation}

The last component performs statistical analysis and data visualisation.
This component must read the sentiment scoring component's output files as input files, to load the dataset of news articles and features extracted by previous components about each article.
At this point, information extracted about each article by previous components must include: source, publication date, and sentiment score, alongside other information.
To combine information about articles from different sources (represented by different files), this component should be capable of reading input from several different input files in a single run, and collate the information about every unique article in each file to a single dataset.

This component must be able to show trends, visualisations, statistical metrics that show how dependent variables (e.g. sentiment score) differ relative to the independent variables (e.g. publication year and source).
The plot types and statistical metrics relevant to this analysis was defined in the end of section \ref{tc-visualisation}.
At a minimum, this component must show how the sentiment score varies relative to publication year and source, and how subsets divided by publication year and source differ in distribution of sentiment scores, using the plots and metrics as defined in section \ref{tc-visualisation}.
Additionally, the component should also show visualisations based on other information extracted by previous components as relevant, such as trends in the term frequency of each keyword and key phrase (as a dependent variable) over time (publication year, as an independent variable).

%% Use cases (a use diagram and list of use case titles, with the full use cases appearing in the appendix).
% Not applicable

%% Results of analysing the requirements to extract information. For example, data modelling to find the data to be stored (ER diagram), views/web pages needed and so on.
\section{Analysis of Requirements} \label{Analysis of Requirements}

% How the requirements influence the design of my project
The implementation design (and additional requirements) of this project are highly influenced by the core requirements (i.e. data collection, filtering, parsing, sentiment scoring, and statistical analysis and visualisation).
The overall design of the pipeline, with isolated components for each sub-problem, stemmed from having largely independent sub-problems that was required in order to perform the data analysis in full, from data collection to analysis and visualisation.
Also, initial prototypes (initial `runs' collecting only a limited number of news articles) showed that data collection took the majority of runtime in the pipeline.
For this reason, the requirement where it should be possible to resume progress on data collection (and not repeat the process for existing data) was added.

The JSON object format of articles for each component's output file were largely defined by the requirements of each component (i.e. the information that must be extracted by each component).
For example, the JSON object format of an article in the data collection component's output file is as follows:
\begin{lstlisting}
{
	"https://www.express.co.uk/comment/columnists/...": {
		"source": "Daily Express",
		"title": "Will she grow out of her stutter?",
		"datetime": "2008-02-12T00:00:00+00:00",
		"section": "comment",
		"subsection": "columnists",
		"text": "..."
	}
}
\end{lstlisting}
Where the URL and `source', `title', `datetime', and `text' fields correspond to requirements defined for the data collection component.

%% The level of detail of the requirements and use cases will depend on the nature of your project. If you are doing a Software Engineering based design and implementation project, then they will need to be done thoroughly. If there is a substantial body of requirements and use cases, then a summary should be given in the chapter, with the full set included in an appendix section.
%% If your project is not Software Engineering oriented, then you still need to describe the requirements you are working to and relevant analysis information. Use cases may not be needed or be relevant.
%% The analysis part of the chapter is what you did to map the requirements information into the first pass design. You can think of analysis as the first stage of design, and the purpose is to show how the requirements were used to inform the design. The length of this chapter depends on the kind of project, but you are typically looking at 5-6 pages.


\chapter{Design and Implementation} \label{Design and Implementation} % 10+ pages (use diagrams, graphs etc. to fill it up)

%% Describe the design of what you have created.
%% Start with the application architecture, giving its overall structure and the components that make up that structure.
\section{Overall Design} \label{Overall Design}

% Components, what each component does, how they interact with each other (show in diagram)
To fulfil the stated requirements in chapter \ref{Requirements and Analysis}, the solution that was decided is a computational pipeline consisting of five main components.
This solution was used to carry out the experiment to collect and analyse online news articles and visualise potential trends in sentiment/opinion.
The results of this experiment is shown in chapter \ref{Results Evaluation}.

The high-level design of the pipeline and its main components were defined as follows:

\vspace{0.5em}
\noindent
\includegraphics[width=\textwidth]{overall-design2.png}

In this experiment, the pipeline was ran once for each source and topic, generating a dataset of all articles for the specified topic from the source with all extracted features, stored in a file.
The `plot' component then loads these files and combines the dataset of all articles for every source within the same topic, and produces various plots to show trends within the topic, and statistical metrics for each possible subset within the topic.

% Input/output format (of JSON files) before/after each stage, how these files are stored (file structure and JSON structure)
Components pass datasets to each other by storing all information it extracts in a JSON format.
The output file of an earlier component is read as the input file for the next component, given the same source and topic.
The format of the output/input files is a collection of JSON objects, where each line consists of a single JSON object, and each JSON object represents a single article.
The JSON object contains a key-value pair for each extracted feature, and each component preserves the previous component's information (key-value pairs) while appending the object with additional key-value pairs for each feature it extracts.
This design ensures that in case computation is interrupted for any reason, the dataset generated by the previous component has already been saved to a file, and computation can restart from the interrupted component (without having to re-run previous components).

%% Include the database or storage representation. (Originally after component description)
\section{Dataset Description} \label{Dataset Description}  % What data have I got

\subsection{Sources} \label{sources}

% sources (Guardian, DM, DE), why
The Internet is an increasingly common medium for news publishers to publish news articles and for consumers to read these articles.
As of 2017, 64\% of British individuals read and/or download online news, newspapers or magazines, a sharp increase from 20\% in 2007 \cite{statista2018share}. 

The Daily Mail (\url{dailymail.co.uk}) and the Guardian (\url{theguardian.com}) are the two most visited news publisher's web sites as of 2016, with a monthly viewership of 11.85 and 10.05 million respectively \cite{statista2018newspaper}, and are the main subjects of this experiment.
Additonally, The Daily Express (\url{express.co.uk}), a slightly smaller newspaper with a monthly viewership of 2.67 million in 2016 \cite{statista2018newspaper}, is also added as a news source in this experiment, to control for source size and explore how the experiment performs with smaller sample sizes.
Furthermore, the Guardian provides an API \cite{guardian} and the Daily Mail and Daily Express provide advanced search tools \cite{daily-mail, daily-express} to query news articles from their websites, which prove useful in their respective data collection component's implementation, although the pipeline would work with any online news source that provides an internal search tool.
Thus, the dataset for this experiment contained news articles from these three news publishers or sources; the Daily Mail, the Guardian, and the Daily Express.

\subsection{Topics, Key Terms, and Query Terms} \label{topics}

% List topics, keywords/phrases, query terms here
A list of topics relevant to disabilities, and a list of key terms for each topic, were compiled for this experiment.
Furthermore, a list of query terms were compiled based on the list of key terms, with words/phrases that could have other meanings (or combinations of multiple common words) removed (unless the word/phrase is very commonly used to refer to the disability, such as `blind' and `mute').
Query terms were used in data collection (where ambiguous terms were removed to reduce off-topic articles), while key terms were used in subsequent components.

The final list of topics relevant to disabilities and people with disabilities, and the key terms and query terms deemed relevant to each topic, were: 

\begin{longtabu} to \textwidth { | X[l] | X[l] | X[l] | } 
	\hline
	Topic & Key Terms & Query Terms \\ 
	\hline
	`disabled' & `disabled', `disability', `handicapped', `cripple', `invalid', `accessible', `ablism', `ableism', `differently abled' & `disabled', `disability', `ablism', `ableism', `differently abled' \\ 
	\hline
	`autism' & `autism', `autistic', `asperger\textbackslash's', `ASD' & `autism', `autistic', `asperger\textbackslash's', `ASD' \\ 
	\hline
	`blind' & `blind', `blindness', `blindism', `visual impairment', `partially sighted', `vision loss' & `blind', `blindness', `visual impairment', `partially sighted', `visually impaired' \\ 
	\hline
	`cerebral palsy' & `cerebral palsy', `spastic' & `cerebral palsy', `spastic' \\ 
	\hline
	`deaf' & `deaf', `deafness', `hearing impaired', `hard of hearing', `hearing loss' & `deaf', `deafness', `hearing impairment', `hard of hearing', `hearing impaired' \\ 
	\hline
	`developmental delay' & `developmental delay', `developmental disability', `developmental disorder', `learning disability', `slow learner', `intellectual disability' & `developmental delay', `developmental disability', `developmental disorder', `learning disability' \\ 
	\hline
	`dyslexia' & `dyslexia', `dyslexic' & `dyslexia', `dyslexic' \\ 
	\hline
	`epilepsy' & `epilepsy', `epileptic', `seizure' & `epilepsy', `epileptic' \\ 
	\hline
	`mental illness' & `mental illness', `mental health', `mental disability', `mental disorder', `mental issue', `brain injured', `brain injury', `brain damaged', `psychological', `psychiatric', `emotional disorder', `behavioural disorder', `retardation', `intellectual disability', `mentally ill', `mentally disabled', `mentally handicapped' & `mental illness', `mental health', `mental disorder', `mental disability', `mentally ill', `mentally disabled', `mentally handicapped' \\ 
	\hline
	`mute' & `mute', `muteness', `mutism', `cannot speak', `difficulty speaking', `synthetic speech', `non-vocal', `non-verbal' & `mute', `muteness', `mutism', `non-verbal' \\ 
	\hline
	`paralysis' & `paraplegic', `quadriplegic', `spinal cord', `paraplegia', `quadriplegia', `paralysed', `paralyzed', `paralysis', `crippled', `leg braces', `wheelchair' & `paraplegic', `quadriplegic', `paraplegia', `quadriplegia', `paralysis' \\ 
	\hline
	`speech impairment' & `speech impairment', `stutter', `speech disability', `speech disorder', `communication disability', `difficulty speaking', `language impairment', `language disorder', `language disability', `speech impediment', `stammer' & `speech impairment', `stutter', `speech disorder', `speech impediment' \\ 
	\hline
\end{longtabu}

% topics: collection of keywords & query terms (keywords minus potentially off topic terms / combinations)
This list of key terms was roughly based on guidelines from the Californian \cite{ca-guideline} and UK \cite{uk-guideline} governments (ignoring whether the term is considered `appropriate' or `inappropriate', as terms labelled `inappropriate' are often still commonly used in the news, and relevant to consider when deciding whether an article is on-topic), with a few additions based on other commonly-used terms found on sampled articles from the Daily Express, the Daily Mail, and the Guardian.

\subsection{Dataset Size} \label{dataset-size}

% Tables for how many articles do I have, for each source, topic, and source+topic, before and after filtering
The dataset is comprised of all articles found online given the query terms defined in section \ref{topics}, published between 2000 to (approximately) end of March 2018.
The size of the initial collected dataset (i.e. all articles collected by the Data Collection component, prior to any further processing) in number of articles, for each source and topic, were:

\begin{center}
	\begin{tabu} to 1.0\textwidth { | X[c] | X[c] | X[c] | X[c] | X[c] | }
		\hline
		Topic & Daily Express & Daily Mail & Guardian & Total \\
		\hline
		Disabled & 16,818 & 24,768 & 30,598 & 72,184  \\
		\hline
		Autism & 988 & 6,035 & 5,780 & 12,803  \\
		\hline
		Blind & 9,467 & 23,616 & 32,307 & 65,390  \\
		\hline
		Cerebral Palsy & 509 & 1,995 & 1,569 & 4,073  \\
		\hline
		Deaf & 6,795 & 20,686 & 8,163 & 35,644  \\
		\hline
		Developmental Delay & 965 & 3,529 & 1,517 & 6,011  \\
		\hline
		Dyslexia & 283 & 938 & 1,980 & 3,201  \\
		\hline
		Epilepsy & 700 & 2,924 & 2,368 & 5,992  \\
		\hline
		Mental Illness & 8,102 & 38,273 & 29,831 & 76,206  \\
		\hline
		Mute & 1,541 & 2,312 & 4,764 & 8,617  \\
		\hline
		Paralysis & 711 & 4,346 & 3,879 & 8,936  \\
		\hline
		Speech Impairment & 1,777 & 2,994 & 1,357 & 6,128  \\
		\hline
		Total & 48,656 & 132,416 & 124,113 & 305,185  \\
		\hline
	\end{tabu}
\end{center}

After filtering, the size of the dataset that was plotted, in number of articles for each source and topic,  were:

% todo double check calculations
\begin{center}
	\begin{tabu} to 1.0\textwidth { | X[c] | X[c] | X[c] | X[c] | X[c] | }
		\hline
		Topic & Daily Express & Daily Mail & Guardian & Total \\
		\hline
		Disabled & 1,852 & 6,035 & 8,524 & 16,411  \\
		\hline
		Autism & 128 & 1,755 & 1,278 & 3,161  \\
		\hline
		Blind & 775 & 3,008 & 3,894 & 7,677  \\
		\hline
		Cerebral Palsy & 25 & 253 & 75 & 353  \\
		\hline
		Deaf & 114 & 747 & 901 & 1,762  \\
		\hline
		Developmental Delay & 5 & 130 & 447 & 582  \\
		\hline
		Dyslexia & 19 & 110 & 281 & 410  \\
		\hline
		Epilepsy & 58 & 740 & 374 & 1,172  \\
		\hline
		Mental Illness & 398 & 6,794 & 8,137 & 15,329  \\
		\hline
		Mute & 24 & 147 & 285 & 456  \\
		\hline
		Paralysis & 53 & 1,003 & 405 & 1,461  \\
		\hline
		Speech Impairment & 57 & 73 & 85 & 215  \\
		\hline
		Total & 3,508 & 20,813 & 24,646 & 48,967  \\ 
		\hline
	\end{tabu}
\end{center}

For the results evaluation, the focus will be on the `disabled' topic, as it has the highest amount of news articles within its subset, and is the most generalisable topic on disabilities (as it refers to the general theme of disabilities and people with disabilities, rather than a specific topic).

\subsection{Limitations} \label{limitations}

% Limitations (e.g. DE has less articles than DM or Guardian, some topics have way less articles), implications for statistical analysis, plotting (e.g. moving averages), etc
As shown in section \ref{dataset-size}, the Daily Express has much fewer articles for any given topic than the Daily Mail or the Guardian.
Some topics, such as cerebral palsy, developmental delay, dyslexia, mute, and speech impairment, are also severely lacking in sample size of articles (post-filter).
In particular, cases where there are less than \textapprox200 articles from a source within a topic were problematic to plot or form statistically-significant conclusions regarding trends (e.g. to compare with other sources), as the distribution of the data is too varied.
Cross-referencing the results in section \ref{Sentiment score: statistical comparison of different sources} with the size of each subset showed that it was difficult to obtain statistically significant conclusions when comparing to subsets with less than \textapprox200 articles.
This means that it is difficult to analyse or compare the Daily Express's subset for topics other than general `disabled', `blind', and `mental illness', due to its relative lack of sample size.

% Also year-on-year limitations (DM only retains articles from ___, DE only retains articles from ___, Guardian only retains articles from ___), and implications (non equal proportion of articles for each year), what it means for analyses 
Another limitation with this experiment is the length of time that each news source retain articles for in their online archive.
Our dataset indicates that by the end of March 2018 (when the data was collected for this experiment), the Daily Express only retains articles from after \textapprox2007, the Daily Mail only retains articles from after \textapprox2010, and the Guardian only retains articles from after \textapprox2000. 
This raises an issue when analysing year-on-year trends, as the proportion of articles' sources within a topic are different in each year, and year-on-year differences may be better explained due to this difference in proportion, rather than an actual trend (see also: section \ref{Sentiment score: plots and trends}).
For this reason, trends over time are only considered to be significant when the trend is consistently repeated for each source's subset (instead of the full dataset for that topic), for all sources with statistically-significant sample size in that topic. 

\section{Components} \label{Components}
%% Give a description of the design of each of the components that make up the architecture.

\subsection{Data Collection} \label{des-data-collection}

% scraper for each source - what needs to be scraper from each article; article scraper AND search engine scraper
The data collection component is composed of two sub-parts.
The first part is a scraper object unique to each supported source (see section \ref{sources}) that defines two functions unique to each source:
\begin{itemize}
	\item A function to return a list of URLs pertaining to news articles related to a specified query term.
		If there are multiple pages, the scraper needs to parse how many search pages exist for the query from the first page, and request each search page.
	\item A function that, given a URL pertaining to an article, parses the article to return the article text and metadata: headline, publication date/time, and publisher.
\end{itemize}
These two functions require separate implementations for each source, as each website has a different HTML structure (which tends to be consistent within the same source).
Thus, the implementations required to parse a list of URLs from the search page and parse text from the article page are different for each source.
The scraper object creates a wrapper for the implementations of these two functions to be injected into the main data collection script, such that the main script will work for any source, with the source-specific implementations of these functions abstracted away.

% crawler: use search engine scraper to get list of article URLs, use article scraper on each URL, store to JSON file line-by-line AND ensure ability to 'save and resume' and not repeat process for articles that already exist in the dataset (by loading a set of URLs that already exist in the dataset beforehand and not calling the scraper for them)
% mention how the requirement to resume on crash/interrupt led to saving a JSON object on each line (to represent each article) in the output so that articles are saved right after they are collected, not after all articles are collected - however, as a result, it violates the JSON standard (only one top-level object per file) - inconsequential but a drawback? regardless
This design allows the use of a single data collection script (the second part of this component) for multiple sources, by having the methods that require separate implementations for each source (finding articles, scraping text and metadata) injected as a dependency to the script.
Furthermore, as defined in the requirements (section \ref{req-data-collection}), it should be possible to re-run the program at a later date to add new articles without sending web requests for articles already in the collection, such that it is possible to `update' the collection and recover collected information in case computation is interrupted.

The data collection script first loads the existing output file for the source and category (if it exists) and loads a list of already-saved URLs.
Then, for each query term in the category, it queries the search page(s) to gather a list of URLs.
It then combines the URLs for all search terms in the category into one list.
For each URL, if the URL is not saved already in the output file, it queries the URL for the article text and metadata, and stores it as a JSON object to an appended line in the output file.
This design allows data collection to be resumed in case computation is interrupted, but with the drawback of technically violating the JSON standard for a valid JSON text to have only one top-level object per file \cite{rfc8259}.

Among all components in the final pipeline (i.e. not including sentiment-scorer comparison), data collection took the longest to compute.
Results from preliminary runs of the experiments indicate that data collection took approximately 0.76 seconds per article for the Daily Express, 0.69 seconds per article for the Daily Mail, and 0.20 seconds per article for the Guardian.
The full dataset of 305,185 articles took roughly a week to collect using a single general-purpose computer (Intel i7-6700HQ CPU @ 2.60GHz, NVIDIA GeForce GTX 1060 GPU) on a 200 Mbit/s (download speed) internet connection.

\subsection{Dataset Filtering} \label{des-filtering}

% What I did
Dataset filtering was implemented by measuring `term frequency ranking', which measures how often does a key term (i.e. keyword or key phrase) appears in a text document relative to other terms.
To compute this metric, the term frequency of every token (word) in the document has to be measured first.
This is implemented using scikit-learn's \cite{Scikit-learn} CountVectoriser, which tokenises a document and converts it into an array of term-frequency values for each token in the document.
This array is then sorted in descending order, and then the position of the keyword in the sorted array (plus one) is recorded as the keyword rank of the document.
A keyword rank of 5, for example, shows that the keyword is the \nth{5} most popular word in the document.

To use keyword rank for filtering documents, a constant threshold value is used to determine whether documents are on-topic or not.
A document is on-topic if the keyword's rank is lower than the threshold value (i.e. the keyword is one of the most frequently used terms in the document), and is off-topic if the keyword's rank is higher than the threshold value (i.e. the keyword is rarely used in the document, relative to other terms).
As the term frequency of the keyword is measured relative to other terms in the document, this metric also normalises for document length.

As in this experiment, each category can contain multiple key terms, slight modifications has to be made during pre-processing (before tokenisation).
All mentions of any keyword or key term in the article are replaced by `KEYWORD\_TOKEN', and the filter measures the rank of `KEYWORD\_TOKEN' instead of any specific term, such that it doesn't matter which term the article uses.
Additionally, stemming is also performed for every word in the document to reduce all words to its root forms, regardless of its word form or tense (e.g. `disability' and `disabilities' are both treated as the same word) using the nltk \cite{Nltk} library.
Furthermore, CountVectoriser also ignores `stop words', or common words in English which do not add topical information, such as `the' and `a'.

% Evaluation of results - how I measured filter effectiveness & determine threshold rank
To evaluate the results of this filtering, the filtering component can be set to print an arbitrarily-sized sample of N articles alongside its predicted label (on-topic/off-topic) and keyword rank.
A test run of 10 articles per topic indicated that for most topics, this simple metric works well in distinguishing between on-topic and off-topic articles.
Manual reading of sampled articles indicated that:
\begin{itemize}
	\item There exists a good of correlation between keyword rank and on-topic/off-topic articles, although the sample size is too low to make statistically-significant conclusions, as increasing the sample size would require manually reading more articles time-consumingly.
	\item In most cases, articles predicted to be `on-topic' are on-topic, and articles predicted to be `off-topic' are off-topic.
\end{itemize}

% Limitations: categories where there are still a lot of off topic sentences
There were, however, topics where this filtering mechanism does not perform well.
As a result, many off-topic articles remain in the filtered dataset.
These topics were `blind', `mute', `paralysis' (`paralysed'), and `speech impairment' (`stutter'), where the keywords can have other meanings irrelevant to the context of people with disabilities.
For example, consider these sampled sentences, taken from articles that were mislabelled on-topic:
\begin{itemize}
	\item ``When Harry met Meghan: How Prince Charles's family friend set up blind date.''
	\item ``Kate Garraway gets flustered as she struggles to mute her ringing phone live on air.''
	\item ``Snooker: Higgins stutters then stages another late comeback.''
\end{itemize}
As term frequency does not distinguish between multiple meanings of words, it is unlikely that this limitation could be solved using any term frequency based approach, or a similar syntactic approach (such as tf-idf and word vectors).

% Alternatives I considered
Several other alternatives were considered to term-frequency ranking.
The conventional approach is by calculating term frequency --- inverse document frequency (tf-idf) \cite{robertson2004understanding, sparck1972statistical}, which builds on term frequency by weighing more `common' words with lower scores and vice versa.
Using tf-idf, terms that appear more frequently in the corpus of all documents are deemed to be `less important' and assigned a lower weight, and vice versa.
However, this approach is unsuitable for this project, due to the selective nature of the dataset (where only articles containing certain query terms are collected). 
This will cause the idf weights of these query terms to be much lower than it should be, because at least one of the query terms would appear on every document in the corpus, highly skewing the document frequency of all query terms in this dataset (relative to within all Daily Mail / Guardian / Daily Express articles).
For this reason, limited evaluation showed that tf-idf ranking performed worse than term-frequency ranking in this filtering task.

A possible improvement to the term-frequency ranking model is by using a supervised machine learning model to classify documents to `relevant' and `irrelevant' (a boolean classification problem).
There are many supervised approaches that are popular for text classification, including Support Vector Matrices (SVM), Na\"{i}ve Bayes (NB), and k-nearest neighbour (kNN) models \cite{khan2010review}.
However, this approach would require labelled data (i.e. a collection of articles with `relevant' and `irrelevant' labels for each topic).

\subsection{Feature Extraction and Rule-based Sentence Matching} \label{des-matching}

% Tokenisation
The next component parses the article text using SpaCy \cite{SpaCy} to extract relevant information required by the sentiment scoring and statistical analysis components.
Initially, the headline is prepended to the article text to create the text document.
SpaCy performs a syntactic analysis of the document, returning a list of `enhanced' tokens (words) which contain additional information parsed by SpaCy, such as part-of-speech tags, syntactic parent/children, known entities, sentence start and end, and root word (lemma).
The component iterates over this list of `enhanced' tokens to extract information that would be useful for statistical analysis:
\begin{enumerate}
	\item Sentences containing a word relating to a keyword or key phrase (using sentence matching), and the number of keywords / key phrases in each sentence.
	\item The frequency of the token or sequence of tokens, for each token that has the same lemma as a keyword, or a sequence of tokens that has the same sequence of lemmas as a key phrase.
	\item Number of relevant sentences in the document (number of occurrences of point 1).
	\item Total number of sentences in the document.
	\item Number of keyword occurrences in the document (number of occurrences of point 2).
	\item Total number of tokens in the document.
\end{enumerate}
Points 3-6 were used to measure a relevance score of each article, however that score is currently unused by the statistical analysis component. 
Future work could expand on this concept.

% Sentence matching
The sentiment scoring component requires all sentences relating to a keyword or key phrase to be extracted from each article.
SpaCy provides a rule-based Matcher tool that returns the indices of tokens (or sequences of tokens) that fulfils a specfic definition (`rule').
This component uses the Matcher tool to find all indices of tokens that has the same lemma as a keyword, or sequences of tokens that has the same sequence of lemmas as a key phrase.
Then, for each token, the component retrieves the full sentence that contains the token index and stores them into a list of relevant sentences, along with the number of keywords / key phrases in each sentence.
These relevant sentences are stored in an array of JSON objects, with each JSON object representing one sentence, such as:
\begin{lstlisting}
"Don't be disabled in spirit as well as physically.\"": {
	"keyword_count": 1
}
\end{lstlisting}
This array is stored as a value within the parent article's JSON object.

Results from running the experiment indicate that this component took approximately 0.37 seconds per article, using a single general-purpose computer (Intel i7-6700HQ CPU @ 2.60GHz, NVIDIA GeForce GTX 1060 GPU).

\subsection{Sentiment Scoring} \label{des-sentiment}

% short paragraph saying 'used open-source libraries' and 'several options were considered and their accuracy analysed on a random sample of sentences, before the two best-performing options were selected to be used for the full dataset, after being optimised for run-time (as running all options on the full dataset will take ~1 minute per article * 51,177 articles = ...)'
The sentiment scoring component implements open-source libraries to measure the `sentiment' (ideally, perceived view of the sentence towards a disability, disabilities, a person with disabilit(y/ies), or people with disabilit(y/ies) as referred by the keyword or key phrase) of news articles.
Sentiment scores are real-valued scores (i.e. a score of 0.0 indicates that a sentence is neutral, a highly positive or highly negative score indicates the sentence has strong positive or negative opinions), capped between -1.0 and 1.0.

Two implementations of the component were developed: 
\begin{itemize}
	\item The first iteration of the sentiment scorer component is used to perform a comparison between several open-source sentiment analysis implementations for this sentiment scoring task.
		This iteration computes sentiment scores of all implemented scorers for every article, and is ran only on an arbitrarily-sized sample of the full dataset, as running all sentiment scorers without optimisation takes roughly \textapprox1 minute per article which is infeasible for the full dataset.
	\item The second iteration of the sentiment scorer component is used in the final pipeline.
		This iteration only computes one sentiment score per each article, and is highly optimised.
		Two iterations of this component were developed: one implementing OpenAI's \cite{OpenAI} model, and another implementing VADER \cite{VADER}.
		In both cases, the runtime of is lower than 0.2 seconds per article.
\end{itemize}

\subsubsection{Comparison of Open Source Implementations} \label{des-sentiment-comparison}

% a 'slow' scorer that takes the first N articles only and scores them with ALL scorers from a list of open-source sentiment scorers, for the purpose of evaluating the performance (accuracy) of these scorers
This component consist of two scripts.
The first script is the sampler, which loads a small sample of articles For each topic and source in the parsed dataset.
It then loads all relevant sentences from the sample of articles to a combined list of relevant sentences from each source and topic.
From these sentences, it selects an arbitrarily-sized sample of relevant sentences for each source and topic (the experiment used 5 sentences per source and topic * 12 topics * 3 sources = 180 sentences).

Every sentence in the sampled are then scored using the 7 open-source sentiment analysis implementations that were explored in section \ref{tc-sentiment}:
\begin{itemize}
	\item VADER \cite{VADER},
	\item xiaohan2012's `twitter-sent-dnn' repository \cite{kalchbrennerACL2014},
	\item kevincobain2000's `sentiment\_classifier' repository \cite{kevincobain},
	\item OpenAI's `generating-reviews-discovering-sentiment' repository \cite{OpenAI},
	\item Stanford CoreNLP \cite{StanfordNLP}, using the stanfordcorenlp package \cite{stanfordcorenlp} was used to start and query a Stanford CoreNLP local server in Python,
	\item TextBlob's PatternAnalyzer \cite{textblob},
	\item TextBlob's NaiveBayesAnalyzer \cite{textblob}.
\end{itemize}

The scores of every sentence in the sample (of 180 sentences in the experiment) is stored in a JSON object similar to:
\begin{lstlisting}
{
	"sentence": "The autism gender trap.",
	"label": "-",
	"sentiment_score_openai": -0.24175840616226196,
	"sentiment_score_vader": -0.3182,
	"sentiment_score_xiaohan": -0.9157504061450769,
	"sentiment_score_kcobain": -0.5,
	"sentiment_score_stanford": -0.5,
	"sentiment_score_textblob": 0.0,
	"sentiment_score_textblob_bayes": -0.9139802175212899
}
\end{lstlisting}
The JSON objects of the sample is then output to a text file, where a user can manually change the `label' fields to either `+' (positive), `-' (negative), `n' (neutral), or `o' (irrelevant/off-topic) to be read by the second script.
To ensure there is no bias in manual labelling, a second output file which contains only sentences and labels (without sentiment scores) were used.

The second script is the analyser.
Given a dataset of labels and sentiment scores (with the format shown above), the analyser plots sentiment score distributions of positive, negative, and neutral sentences in seven histograms (one for each sentiment scorer).
Additionally, it also computes these statistics for each sentiment scorer:
\begin{itemize}
	\item Mean positive: mean sentiment score for all positive-labelled sentences
	\item Mean neutral: mean sentiment score for all neutral-labelled sentences
	\item Mean negative: mean sentiment score for all negative-labelled sentences
	\item True positive: count of positive-labelled sentences with sentiment score $>$ 0.0
	\item False positive: count of negative-labelled sentences with sentiment score $>$ 0.0 
	\item True negative: count of negative-labelled sentences with sentiment score $\le$ 0.0
	\item False negative: count of positive-labelled sentences with sentiment score $\le$ 0.0
	\item Accuracy: (True positive + false positive) / (count of all positive or negative sentences)
\end{itemize}
The results of this sentiment scorer comparison is documented in section \ref{Comparison of sentiment scorers}.

\subsubsection{Final Implementation} \label{des-sentiment-final}

% After analysis, decided on running only two 'final' sentiment scorers for the whole dataset, using a separate function that computes only one sentiment score, optimised for performance (speed / runtime)
The results of the sentiment scorer comparison, shown in section \ref{Comparison of sentiment scorers}, show VADER \cite{VADER} and OpenAI's \cite{OpenAI} models as the two best sentiment scoring tools for this domain of measuring the perceived `sentiment' of a sentence towards a disability, disabilities, a person with disabilit(y/ies), or people with disabilit(y/ies).

Unlike the iteration used for sentiment scorer comparison, this iteration only computes one sentiment score per each article, and is highly optimised.
Two versions of this iteration were developed: one implementing OpenAI's model, and another implementing VADER, to measure sentiment scores.
In the final pipeline used for the experiment, only the version using VADER was used, as VADER's scores was shown to be better at displaying trends separating different subsets than the OpenAI model's scores (refer to section \ref{Comparison of sentiment scorers}).

Within the pipeline, this component's task is to compute sentiment score information for every relevant sentence, and append that information on each sentence's JSON object:
\begin{lstlisting}
"Don't be disabled in spirit as well as physically.\"": {
	"keyword_count": 1,
	"sentiment_score": 0.4215
}
\end{lstlisting}
Once the scores of all sentences had been measured, this component also computes the sentiment score of each article, which is defined as the weighted average of sentiment scores for all relevant sentences contained within the article:

\begin{center}
	$Article's \ sentiment \ score = \dfrac{\Sigma (sentiment \ score * number \ of \ key \ terms) for \ each \ sentence}{Total \ number \ of \ keyword \ occurrences \ in \ the \ article}$
\end{center}

% OpenAI: 0.16 seconds, VADER: 0.0031 seconds. Expand more in the subsubsection.
This iteration is highly optimised for runtime: it only computes one sentiment score for each sentence, using one sentiment model, instead of loading and analysing every sentence with all seven sentiment models as in section \ref{Comparison of sentiment scorers}.
For the version that implements OpenAI's model, it is further optimised by `batching' the call to the sentiment model: instead of calling `openai\_model.transform()' for every sentence, it builds a corpus (list) of all sentences in all articles from the loaded file (where each file represents all articles from a source for a topic) and calls `openai\_model.transform()' only once on the full corpus.
Results from running the experiment indicate that sentiment scoring took approximately 0.16 seconds per article using the OpenAI model, and 0.0031 seconds using VADER, using a single general-purpose computer (Intel i7-6700HQ CPU @ 2.60GHz, NVIDIA GeForce GTX 1060 GPU).

\subsection{Statistical Analysis and Visualisation} \label{des-visualisation}

This component is responsible for statistical analysis and data visualisation, plotting graphs and saving statistical information for each topic.
At this point, each article's JSON representation includes information about its source, publication date, sentiment score, and other information extracted by previous components.
However, up to this point, the dataset of news articles belonging to each source and each topic were kept in separate files.
Instead of loading just one file per execution as in previous components, this component had to load all files that relates to a specified topic from every source, and combines the datasets of all articles from every source that relates to the specified topic.
This is necessary to compare different sources in a plot and perform statistical comparisons.
(Refer to the diagram in section \ref{Overall Design} for more details on the pipeline design for each topic.)

Most of this component is concerned with plotting and measuring how sentiment score varies when the publication source and date/year of publication is varied.
For easier data processing, it loads the publication date, sentiment score, and source information of each article as rows in a numpy array matrix, sorted by publication time.

The component then uses the matrix to produce the following plots using matplotlib \cite{Matplotlib}:
\begin{itemize}
	\item A scatter plot of publication date (X) vs sentiment (Y), coloured based on source (Z).
		% (A scatter plot displays data as a collection of points, where the position of points in axis X and axis Y correspond to its value over two separate variables)
	\item A regular histogram showing the number of news articles in the dataset for each year.
		% (A histogram is a plot showing the 'frequency' (number) of values in each interval of a variable, shown as rectangles with varying heights similar to a bar graph)
	\item Two-dimensional histograms showing:
		\begin{itemize}
			\item The number of news articles for each year (X) and source (Y) in the dataset.
			\item The number of news articles for each year (X) and `sentiment range' (Y) in the dataset.
			(`sentiment range': news articles are grouped together based on sentiment score with intervals of 0.1; thus, for example, -0.1--0.0, 0.0--0.1, and 0.5--0.6 are sentiment ranges)
			\item The number of news articles for each source (X) and `sentiment range' (Y) in the dataset.
		\end{itemize}
		% (A two-dimensional histogram is similar to a regular histogram, except both X and Y axes are intervals of two separate variables, and frequency is shown in a colour scale instead of height)
	\item A line graph of the moving average of sentiment score (Y) over time (X), with separate lines for each source.
		The moving average is defined as the mean of sentiment scores for W previous articles up to the current article, where W is the moving average window size.
		In this experiment, W = (no. of articles in the topic) / 10, with a lower cap of 50 and an upper cap of 500.
	\item A violin plot \cite{hintze1998violin} and box-and-whiskers plot \cite{tukey1977exploratory} showing the sentiment score distributions (including mean, upper and lower quartiles, and density plots) for each data subset, separated by source.
	\item A violin plot showing the sentiment score distributions for each data subset separated by source and publication year, with intervals of 2 years, separated to two plots: one for 2000--2009 and one for 2010--2019.
\end{itemize}
These plots are arranged in a 3x3 grid and saved to an output file, unique to each topic. 
Refer to section \ref{tc-visualisation} for a further description of each plots' usage.

To quantify the distribution of each data subset, the component computes the mean, standard deviation, and total count of articles for the full dataset and each possible subset of the data (i.e. a set of all articles for every source, a set of all articles for every year, and a set of all articles for every year and every source). 
These metrics are saved to an output text file, unique to each topic.

It also attempts to compare whether the sentiment scores of articles published by a source is significantly higher than the sentiment scores of articles published by another source. 
The Mann-Whitney $U$ Test \cite{mann1947test} is a non-parametric statistical test that measures the null hypothesis that ``given a randomly-selected value from a distribution, and another randomly-selected value from another distribution, the first value is equally likely to be less or greater than the second value.''
If the null hypothesis holds true, then there are no statistically significant difference between the two distributions.
The Mann-Whitney $U$ Test returns the $U$ statistic and a $p$-value between 0 and 1, which corresponds to how likely the null hypothesis is to be true; the null hypothesis is rejected if the $p$-value is lower than 0.05.
This test is performed to compare the subset of all articles published by each source to the subset of all articles published by each other source; for all articles within a topic, and for all articles published in each year within a topic.
These test results are saved to the same output text file.

As well as information relating to sentiment score, this component also attempts to plot trends regarding the usage of key terms (used to describe disabilities or people with disabilities) over time.
This information is collected from the frequencies of tokens with the same lemma as a key term, as mentioned in point 2 of section \ref{des-matching}.
The tokens are first stemmed with a custom stemmer: unlike NLTK's stemmer or SpaCy's lemmatiser as used in previous components, this stemmer only removes plurals (-s, -es) and tenses (-ed, -ing), but does not fully reduce words to its base word form (e.g. `illness' and `illnesses' are equivalent, but `mental' and `mentally' are still separate words).
Then, the dataset is split into smaller subsets based on publication year and source.
For each subset, the term frequencies for all articles within the subset are averaged, to compute a measure of average term occurrence (per article):

\begin{center}
	$Average \ term \ occurrence = \dfrac{\Sigma (term \ frequency \ in \ article) \ for \ each \ article}{number \ of \ articles \ in \ subset}$
\end{center}

For each term, the annual average term occurrences are plotted in a line graph of average term occurrences (Y) over publication year (X).
Each unique term has its own plot, with separate lines for each source. 
These plots are then saved to an output file, separate from the sentiment score plots, also unique to each topic. 

%% Provide implementation details as necessary.

%% As with other chapters, the structure and contents of this chapter will depend on the nature of your project, so the list above is only a suggestion, not a fixed requirement.

%% Find an ordering and form of words so that the design is clear, focusing on the interesting design decisions. For example, what were the alternatives, why select one particular solution? You have a limited number of pages so be selective about details. Also remember that someone (your examiners!) has to read this so don’t overwhelm them with intricate descriptions of everything that only you can follow – but do make sure the key details of the solution are in place. Use appropriate terminology and demonstrate that you have a good understanding of the Computer Science principles involved.

%% You can use diagrams and screen shots to help explain the design but don’t overuse them. Diagrams and screen shots should add information, not duplicate what is written in the text, and definitely avoid page after page of diagrams as this will disrupt the flow of your text. Where relevant, UML diagrams can certainly be used but, again, don’t flood the chapter with diagrams. Additional diagrams can always be included in an appendix section. It is not the case that a full set of UML diagrams must be provided for a software development project, and they shouldn’t be added in the belief that there must be UML diagrams to do a good project. Think about what you need to communicate and use UML diagrams if and when they fit the need.

%% It may be useful to include sections of code to highlight how a particular algorithm is implemented or how an interesting programming problem was solved. However, avoid lengthy sections of code, as this can also disrupt the flow of the text. Also make sure that your code fragments are readable, easy to follow and properly laid out. It may be better to use pseudo-code rather than actual code, especially when describing an algorithm. If you need to make use of longer sections of code, you can put the code in the appendix and reference it from the text.

%% An alternative way to organise the content of both this chapter and the preceding one, suitable for some projects, is to have a sequence of chapters or sections for each major iteration of the project. This allows the progression of the project to be shown, with each iteration building on the last, and the opportunity for interesting discussion about the decisions that needed to be made.

%% This is a core chapter in your report and will usually be quite substantial, 10 pages or more.


\chapter{Results Evaluation} \label{Results Evaluation}  % 2-4 pages (can be more if needed, probably more with graphs and tables)

\section{Focused Topic} \label{Focused topic}
% This chapter will focus on the 'disabled' topic ... 
% Before going into sections, mention that although I have researched 12 topics relating to specific disabilities, this topic is chosen as an example for the resuls evaluation chapter as it is the most generic.
Although the pipeline was run on a list of twelve disability-related topics (as defined in section \ref{topics}), this results evaluation will mainly focus on the `disabled' topic.
The `disabled' topic consist of articles that relates to the key terms: `disabled', `disability', `handicapped', `cripple', `invalid', `accessible', `ablism', `ableism'.
This topic was chosen as it had the largest sample of news articles within the dataset (see also: section \ref{dataset-size}), and it is the most generalisable topic (as it refers to the general theme of disabilities and people with disabilities, rather than a specific topic).
This focus on one topic for the whole of Chapter \ref{Results Evaluation} helps keep the numbers and results being discussed consistent throughout Chapter \ref{Results Evaluation}.
Furthermore, the full result plots for all topics will be available in the Appendix.

At several points in this chapter, results from other topics will also be discussed where it would add to the discussion.
Information from other topics will be explicitly mentioned (e.g. ``For topics other than `disabled','') such that the reader understands where the data does not refer to the `disabled' topic.

% Explain the 'disabled' dataset first: provide a table with no. of articles for each source in each year.
Within the `disabled' topic, the (post-filtering) sample size that was obtained for each year between 2000 and 2018 is as follows:

\begin{longtabu} to 1.0\textwidth { | X[c] | X[c] | X[c] | X[c] | X[c] | }
	\hline
	Year & Daily Express & Daily Mail & Guardian & Total \\
	\hline
	2000 & 0 & 0 & 504* & 504  \\
	\hline
	2001 & 0 & 0 & 294 & 294  \\
	\hline
	2002 & 0 & 0 & 334 & 334  \\
	\hline
	2003 & 0 & 2 & 330 & 332  \\
	\hline
	2004 & 0 & 2 & 387 & 389  \\
	\hline
	2005 & 0 & 0 & 381 & 381  \\
	\hline
	2006 & 0 & 0 & 338 & 338  \\
	\hline
	2007 & 51 & 0 & 424 & 475  \\
	\hline
	2008 & 86 & 0 & 424 & 510  \\
	\hline
	2009 & 175 & 0 & 352 & 527  \\
	\hline
	2010 & 157 & 173 & 365 & 695  \\
	\hline
	2011 & 166 & 370 & 607 & 1,143  \\
	\hline
	2012 & 238 & 516 & 822 & 1,576  \\
	\hline
	2013 & 133 & 473 & 638 & 1,244  \\
	\hline
	2014 & 140 & 846 & 544 & 1,530  \\
	\hline
	2015 & 177 & 1,095 & 526 & 1,798  \\
	\hline
	2016 & 260 & 1,128 & 640 & 2,028  \\
	\hline
	2017 & 220 & 1,094 & 506 & 1,820  \\
	\hline
	2018** & 49 & 336 & 108 & 493  \\
	\hline
	Total & 1,852 & 6,035 & 8,524 & 16,411  \\ 
	\hline
\end{longtabu}
* 2000 data also includes a small amount of articles published before the year 2000.

** 2018 data is incomplete and would only include articles up to (approximately) end of March.

\section{Comparison of Sentiment Scorers} \label{Comparison of sentiment scorers}
% Short re-explanation of how these were compared, by taking a sample of 180? sentences, and manually labelling them, then plotting the sentiment-score distributions of 'positive' and 'negative' sentences for each scorer (summary of 4.3.4.1) \ref{des-sentiment-comparison}
For this comparison, an evenly-distributed sample of relevant sentences (i.e. sentences referring to disabilities or people with disabilities) were taken from the filtered dataset of every topic. 
The sample contains 5 sentences from each source (Daily Express, Daily Mail, Guardian) and each of the 12 topics (as defined in section \ref{topics}). % 15 sentences per topic?
The sample contains 180 sentences in total. % 180 = 5*3*12

Each sentence in the sample are then manually labelled to: positive, negative, neutral, or irrelevant.
These labels are measured by the perceived `sentiment' of a sentence towards a disability, disabilities, a person with disabilit(y/ies), or people with disabilit(y/ies), or `irrelevant' if it did not refer to any disability-related topic.

The sentiment score distributions of positive (blue), negative (red), and neutral (green) sentences for each topic were plotted as follows:
% results for how open-source sentiment scorers compare: plots

\noindent
\includegraphics[width=\textwidth]{eval.png}

From these plots, it is clear that `vader' \cite{VADER} and `openai' \cite{OpenAI} are the two most encouraging sentiment scorers on this domain, as they show a clear distinction between the distributions of `positive' and `negative' labels (although with some overlap near the centre), while other scorers produce plots where the values are all over the place.

The means of these distributions, the count of `positive'-labelled sentences with a positive ($>$0, true positive) and negative ($\le$0, false negative) sentiment scores, the count of `negative'-labelled sentences with a positive ($>$0, false positive) and negative ($\le$0, true negative) sentiment scores, and accuracy (defined in section \ref{des-sentiment-comparison} as (true positive + true negative) / (all positive + all negative)) were also measured.
Refer to section \ref{des-sentiment-comparison} for a formal definition of these metrics.

The values of these metrics for each sentiment scorer are shown below: 

% mean positive/neutral/negative, true/false positives/negatives, accuracy (in a table)
\vspace{0.5em}
\noindent
\begin{tabu} to 1.0\textwidth { | X[c] | X[c] | X[c] | X[c] | X[c] | }
	\hline
	Implementation & Mean Positive & Mean Neutral & Mean Negative & Accuracy* \\
	\hline
	VADER \cite{VADER} & 0.327 & -0.048 & -0.313 & 0.789 \\
	\hline
	XiaoHan \cite{kalchbrennerACL2014} & -0.071 & -0.306 & -0.681 & 0.621 \\
	\hline
	Kevin Cobain's \cite{kevincobain} & 0.348 & 0.210 & -0.116 & 0.621 \\
	\hline
	OpenAI's \cite{OpenAI} & 0.301 & 0.128 & -0.025 & 0.758 \\
	\hline
	Stanford CoreNLP \cite{StanfordNLP} & -0.196 & -0.394 & -0.398 & 0.568 \\
	\hline
	TextBlob \cite{textblob} (Pattern) & 0.086 & 0.007 & -0.030 & 0.663 \\
	\hline
	TextBlob \cite{textblob} (Na\"{i}ve Bayes) & 0.572 & 0.508 & 0.123 & 0.653 \\
	\hline
\end{tabu}

\vspace{0.2em}
*Binary classification accuracy (accuracy of scores for `positive' and `negative' labels in the sample, disregarding `neutral' or `irrelevant' labels)

`neutral' or `irrelevant' labels were disregarded in the accuracy measurement as they do not have an expected value (whereas `positive' is accurate if value $>$0, and `negative' is accurate if value $\le$0)
For the following measurements, `neutral' and `irrelevant' labels were also disregarded:

\vspace{0.5em}
\noindent
\begin{tabu} to 1.0\textwidth { | X[c] | X[c] | X[c] | X[c] | X[c] |}
	\hline
	Implementation & True Positive & False Positive & True Negative & False Negative \\
	\hline
	VADER \cite{VADER} & 35 & 4 & 40 & 16 \\
	\hline
	XiaoHan \cite{kalchbrennerACL2014} & 22 & 7 & 37 & 29 \\
	\hline
	Kevin Cobain's \cite{kevincobain} & 35 & 20 & 24 & 16 \\
	\hline
	OpenAI's \cite{OpenAI} & 46 & 18 & 26 & 5 \\
	\hline
	Stanford CoreNLP \cite{StanfordNLP} & 13 & 3 & 41 & 38 \\
	\hline
	TextBlob \cite{textblob} (Pattern) & 32 & 13 & 31 & 19 \\
	\hline
	TextBlob \cite{textblob} (Na\"{i}ve Bayes) & 43 & 25 & 19 & 8 \\
	\hline
\end{tabu}
\vspace{0.5em}
	
These results from the 180-article sample indicate that VADER \cite{VADER}, followed by OpenAI's model \cite{OpenAI}, as the two best-performing open-source sentiment scoring tools for this domain, as shown by the accuracy metric.
Although these results are not conclusive (given the small sample size of sentences, as it was necessary to manually label each sentence in the sample), it is a sufficient indicator of which sentiment scorers would perform better in predicting correct labels, and therefore should be chosen for the final pipeline and experiment.
(Additionally, it shows that the sentiment score distribution generated by OpenAI is slightly skewed positive, while the distribution generated by VADER is slightly skewed negative, based on the mean values and ratio between false positives and false negatives).

To prove whether these accuracy metrics are relevant for the experiment, both VADER and OpenAI's sentiment scorer implementations were implemented in the final pipeline.
Then, the 500-article moving average sentiment score of both scorers were plotted for the topic `disabled':

% show comparison of 'disabled' moving average plot (VADER vs OpenAI)
\noindent
\includegraphics[width=0.5\textwidth]{vader.png}
\includegraphics[width=0.5\textwidth]{openai.png}
(Left = VADER, Right = OpenAI's model)
\vspace{0.5em}

These plots show that the version implementing VADER is more consistent in distinguishing between different sources, with a smoother trend line; while the line produced by the version implementing OpenAI's model has higher randomness; despite using the same moving average window size (500 articles) in both plots.

% discuss about how VADER, although much simpler than other models, is more generalisable than stat-ML/NN models trained on different domains ("These results came as a surprise as...")
These results came as a surprise, due to the simplicity of VADER's rule-based model, relative to other implemented models (refer to section \ref{tc-sentiment} or \ref{des-sentiment-comparison} for a list and short description of these sentiment scorer implementations) based on supervised machine learning or neural network approaches.
The suspected reason behind this is that because these supervised machine learning and neural network models were trained on data from other domains (mainly tweets, movie reviews, or Amazon reviews), and the parameters learned by the model does not necessarily translate well to this domain (sentences from news articles referring to disabilities or people with disabilities). 
Meanwhile, VADER's simpler rule-based model is more generalisable, as it simply checks for a set of rules to determine general positive/negative text (instead of using parameters learned from training data).

It is likely that a supervised machine learning or neural network model, trained on an adequately large labelled dataset (likely at least tens or hundreds of thousands of sentences) from this domain, could strongly out-perform VADER, given that the accuracy of sentiment scores from OpenAI's implementation (based on a neural network model) was very close to VADER's, despite the model being trained on a different domain (82 million Amazon product reviews). 
However, such a large labelled dataset was infeasible for the scope of this experiment.

% Mention how only results from VADER will be considered relevant from this point
For the following sections, all sentiment score results mentioned are those scored by VADER's implementation \cite{VADER}.
Additionally, news articles with a VADER sentiment score of 0.0 were excluded from the plotted dataset, as they are presumed to be non-opinionated and thus irrelevant in sentiment measurement.

\section{Sentiment Score: Plots and Trends} \label{Sentiment score: plots and trends}
% graphs, tables (mean std dev) relating to sentiment
For the `disabled' topic, the VADER sentiment scores of all articles within the dataset (16,411 articles in total within the topic) were plotted with regards to their publication date and source of article (Daily Express, Daily Mail, Guardian) in several visual representations (as defined in section \ref{des-visualisation}).

\begin{center}
	\includegraphics[height=0.5\linewidth]{row-1-col-1.png}
\end{center}

The first plot is a scatter plot of publication date (X) vs sentiment scores (Y), coloured based on publication source (Z). 
Due to the large dataset size, the scatter plot does not provide much information with regards to distributions and trends, although it shows how the Daily Express (blue-green) only retains articles from after \textapprox2007, the Daily Mail (dark blue) only retains articles from after \textapprox2010, and the Guardian (yellow) only retains articles from after \textapprox2000, as mentioned in section \ref{limitations}.
A moving average line (with a window size of 500) of sentiment scores over time of all articles (regardless of source) is drawn over the scatter plot.

\begin{center}
	\includegraphics[width=0.5\linewidth]{row-1-col-3.png}
\end{center}

A histogram of the number of articles in each year was plotted to show the distribution of articles over time in the dataset. 
This do not necessarily reflect how many articles was published regarding the topic for every year, as news publishers often do not retain all historical articles and the amount of retained articles tend to be higher in more recent years, and vice versa.
That said, a spike in the number of articles in the dataset was visible in 2012, which correspond to increased media coverage of disabilities and people with disabilities around the 2012 London Paralympics.

\begin{center}
	\includegraphics[width=0.5\linewidth]{row-1-col-2.png}
\end{center}
\includegraphics[width=0.5\linewidth]{row-2-col-1.png}
\includegraphics[width=0.5\linewidth]{row-2-col-2.png}

Three two-dimensional histograms that show the distribution of articles for:
\begin{itemize}
	\item The number of news articles for each year (X) and source (Y) in the dataset.
	\item The distribution of sentiment scores (Y) for each year (X).
	\item The distribution of sentiment scores (Y) for each source (X).
\end{itemize}
were plotted. 
The year-source plot show that the dataset consist almost entirely of Guardian articles for the years 2000--2006. In 2007 Daily Express articles and in 2010 Daily Mail articles starts to appear, although Guardian articles still predominate between 2007--2013, until in 2014 where Daily Mail articles start to outnumber Guardian articles by roughly 2:1.

The year-sentiment plot show that the sentiment distribution of most news articles lie somewhere between 0.0 and 0.3 from 2000--2009, and between -0.3 and 0.3 (but with more variance/outliers) from 2010--2018. 
It also showed a slight `drop' in the sentiment distribution (or an increase of news articles with negative sentiment scores) around 2016.

The source-sentiment plot show (roughly) that the Guardian has a higher mean and less variance than the Daily Mail, and the Daily Express is barely visible due to the lower sample size of Daily Express articles in the dataset.
However, the uneven sample size (i.e. more Guardian articles vs Daily Mail or Daily Express articles over the full dataset) makes this visual representation hard to compare. The violin \cite{hintze1998violin} and box-and-whiskers plot \cite{tukey1977exploratory} show a better representation of this source-sentiment data:

\begin{center}
	\includegraphics[width=0.5\linewidth]{row-3-col-1.png}
\end{center}

The violin and box-and-whiskers plot show that the Guardian's mean sentiment score (0.074) is higher than the Daily Mail's (-0.104) or the Daily Express's (-0.060).
Furthermore, the violin plot component show that the Guardian's distribution is more `compact' around the mean (std. dev. = 0.363), the Mail has a slightly higher variance (std. dev. = 0.386), and the Express has the highest variance (std. dev. = 0.447). A version that shows violin plots for each 2-year period were also plotted (green = Guardian, blue = Daily Express, orange = Daily Mail):

\begin{center}
	\includegraphics[width=0.4\linewidth]{row-3-col-2.png}
	\includegraphics[width=0.4\linewidth]{row-3-col-3.png}

	\includegraphics[width=0.5\linewidth]{row-2-col-3.png}
\end{center}

From the year-sentiment plot showed that there is an apparent trend of declining sentiment scores over time. 
However, the moving average line plot (where each data point shows the mean sentiment score of 500 consecutive articles, and the last article's date) show that this is not necessarily the case. 
When only taking into account articles from the same source, the moving average sentiment score stays roughly consistent.
However, the decreasing trend in `all' is likely better attributed to the (gradually) decreasing proportion of Guardian articles and the (gradually) increasing proportion of Daily Mail articles in the dataset.

For topics other than `disabled', the majority show a similar constant trend for articles within the same source, although this is not always the case. The topic `autism', for instance, show a positive trend in sentiment scores year-on-year:

\begin{center}
	\includegraphics[width=0.5\linewidth]{autism.png}
\end{center}

Note also that the Daily Express is not plotted for this topic, due to the sample size being too small (128 articles), lower than the moving average window (316 articles on this topic); a common occurrence for most other topics (as defined in \ref{topics}).

\section{Sentiment Score: Statistical Comparison of Sources} \label{Sentiment score: statistical comparison of different sources}

% Mann-Whitney U analysis (re-define it again, what it means for statistical significance etc)
To test whether the differences between the distributions of each source are statistically significant, the Mann-Whitney $U$ Test \cite{mann1947test} was invoked.
The Mann-Whitney $U$ Test is a mathematical function that tests the null hypothesis that given a randomly-selected value from a distribution, and a second randomly-selected value from a second distribution, the first value is equally likely to be less than or greater than the second value.
If the null hypothesis holds true, then there is no statistically significant difference between the two distributions.
The Mann-Whitney $U$ Test function returns a $p$-value score, which is a measure of the probability that the null hypothesis is correct.
In this experiment, two distributions are considered significantly different if the $p$-value returned by the Mann-Whitney $U$ Test is lower than 0.05.

% Table of Mann-Whitney U scores for each topic and source comparisons, overlap with n_articles for each source
Below is a table of $p$-values from Mann-Whitney $U$ Test results for every source combination in the dataset, for every topic (including topics other than `disabled'):

\noindent
\begin{tabu} to 1.0\textwidth { | X[c] | X[c] | X[c] | X[c] | }
	\hline
	Topic & Guardian \textgreater\space Daily Mail & Guardian \textgreater\space Daily Express & Daily Express \textgreater\space Daily Mail  \\
		\hline
	Disabled & \textbf{2.66 * 10\textsuperscript{-167}} & \textbf{2.29 * 10\textsuperscript{-35}} & \textbf{0.000108}  \\
	\hline
	Autism & \textbf{3.00 * 10\textsuperscript{-12}} & 0.137 & 0.184  \\
	\hline
	Blind & 0.335 ** & \textbf{1.62 * 10\textsuperscript{-9}} & \textbf{6.57 * 10\textsuperscript{-9} **} \\
	\hline
	Cerebral Palsy & 0.469 & 0.156 ** & 0.169  \\
	\hline
	Deaf & \textbf{6.30 * 10\textsuperscript{-10}} & 0.141 & 0.0833  \\
	\hline
	Developmental Delay & 0.383 & \textbf{0.00179} & \textbf{0.00326 **}  \\
	\hline
	Dyslexia & 0.0740 & 0.362 & 0.464 **  \\
	\hline
	Epilepsy & \textbf{3.39 * 10\textsuperscript{-5}} & 0.0579 & 0.322 **  \\
	\hline
	Mental Illness & \textbf{3.54 * 10\textsuperscript{-76}} & \textbf{1.51 * 10\textsuperscript{-9}} & 0.102 **  \\
	\hline
	Mute & \textbf{6.78 * 10\textsuperscript{-5}} & \textbf{0.0311} & 0.346 ** \\
	\hline
	Paralysis & \textbf{2.73 * 10\textsuperscript{-5}} & \textbf{0.00909} & 0.0827 **  \\
	\hline
	Speech Impairment & 0.235 & \textbf{0.0200} & 0.0546 **  \\
	\hline
\end{tabu}
** Indicates where the reverse assumption is true (e.g. Daily Mail \textgreater\space Daily Express instead of Daily Express \textgreater\space Daily Mail)

This data shows that a statistically-significant distinction can be proven between the sentiment score distributions of two sources for 17 / 36 of cases.
(Cases where there exists a distinction between the two distributions are highlighted in bold)
In particular, Guardian \textgreater\space Daily Mail is true for 7 / 12 cases, Guardian \textgreater\space Daily Express is true for 7 / 12 cases, Daily Express \textgreater\space Daily Mail is true for 1 / 12 cases, and Daily Mail \textgreater\space Daily Express is true for 2 / 12 cases.
Comparing these values to the size of each dataset (section \ref{dataset-size}), Lower $p$-values tends to correlate well with larger sample sizes (i.e. the higher the sample size of both distributions, the higher the chance that there exist a statistically-significant difference).

% do year on year analysis on disabled topic
For the `disabled' topic, the Mann-Whitney $U$ Test was also performed to compare between Daily Mail, Daily Express, and The Guardian for each year's subset between 2007 and 2018 (Before 2007, there were not enough non-Guardian articles in the dataset to make a comparison).
The $p$-values of these comparisons are shown below:

\noindent
\begin{tabu} to 1.0\textwidth { | X[c] | X[c] | X[c] | X[c] | }
	\hline
	Topic & Guardian \textgreater\space Daily Mail & Guardian \textgreater\space Daily Express & Daily Express \textgreater\space Daily Mail  \\
	\hline
	2007 & N/A & \textbf{5.24 * 10\textsuperscript{-6}} & N/A  \\
	\hline
	2008 & N/A & 0.0739 & N/A  \\
	\hline
	2009 & N/A & \textbf{2.24 * 10\textsuperscript{-5}} & N/A  \\
	\hline
	2010 & \textbf{6.00 * 10\textsuperscript{-7}} & \textbf{6.89 * 10\textsuperscript{-6}} & 0.360 **  \\
	\hline
	2011 & \textbf{5.73 * 10\textsuperscript{-16}} & \textbf{9.73 * 10\textsuperscript{-5}} & \textbf{0.0355}  \\
	\hline
	2012 & \textbf{4.44 * 10\textsuperscript{-17}} & \textbf{0.00371} & \textbf{0.000739}  \\
	\hline
	2013 & \textbf{5.03 * 10\textsuperscript{-16}} & \textbf{0.00351} & 0.0827  \\
	\hline
	2014 & \textbf{1.27 * 10\textsuperscript{-9}} & \textbf{0.00208} & 0.321  \\
	\hline
	2015 & \textbf{6.87 * 10\textsuperscript{-17}} & \textbf{0.00855} & \textbf{0.00718}  \\
	\hline
	2016 & \textbf{4.63 * 10\textsuperscript{-26}} & \textbf{4.74 * 10\textsuperscript{-6}} & 0.0576  \\
	\hline
	2017 & \textbf{3.39 * 10\textsuperscript{-13}} & \textbf{0.00244} & 0.0867  \\
	\hline
	2018* & 0.122 & 0.120 & 0.328  \\
	\hline
\end{tabu}

* 2018 data is incomplete and would only include articles up to (approximately) end of March.

** Indicates where the reverse assumption is true (e.g. Daily Mail \textgreater\space Daily Express instead of Daily Express \textgreater\space Daily Mail)

\section{Key Terms: Plots and Trends} \label{Key terms: plots and trends}

Apart from sentiment score analyses, NLP approaches can also be applied to extract various other syntax-based features and analyse trends based on them.
Features based on counting words (term frequency) are a mainstay of NLP research.
Besides sentiment scores, the pipeline also extracts the term frequency of all terms that matches a key term's lemma from each news article (refer to section \ref{des-matching} for details).
This information is used to plot trends on keyword usage over time.

A weak custom stemmer that only stems plural and past/future tense forms (but not other suffixes) allows the grouping of equivalent terms together regardless of context, but without losing additional meaning from affixes.
For example, `illness' and `illnesses' are equivalent, but `ill' and `illness' are still separate words.

For each term, a measure of average term occurrence, which measures the expected number of occurrences of a term in an article for a given year, is computed for each year between 2000 and 2018.
Refer to section \ref{des-visualisation} for a formal definition of this metric.

With this approach, we measured the year-on-year average term occurrence trends of each key term in the `disabled' topic, and plotted the results in line graphs (one for each key term):

\noindent
\includegraphics[width=0.5\linewidth]{terms1.png}
\includegraphics[width=0.5\linewidth]{terms2.png}

\noindent
\includegraphics[width=0.5\linewidth]{terms3.png}
\includegraphics[width=0.5\linewidth]{terms4.png}

\noindent
\includegraphics[width=0.5\linewidth]{terms5.png}
\includegraphics[width=0.5\linewidth]{terms6.png}

\noindent
\includegraphics[width=0.5\linewidth]{terms7.png}
\includegraphics[width=0.5\linewidth]{terms8.png}

\begin{center}
	\includegraphics[width=0.5\linewidth]{terms9.png}
\end{center}

% key terms analysis: most of them stay constant over the period (probably because 18 years data is quite short for cultural shifts), but mention the few where visible trends can be seen + show graphs. Also smaller sample sizes.
These line plots show that, for most key terms, the average term occurrence trends mostly stay constant year-on-year for the same publisher.
While this exact approach has been successfully used to detect cultural changes in British media \cite{lansdall2017content}, in this case the sampled time period is much shorter for significant linguistic changes to have occurred (18 vs 150 years).

In this case, increasing/decreasing year-on-year trends are only visible for a select few terms.
The usage of the terms `invalid' and `handicap(ped)', for example, show a rapidly decreasing trend between 2000 and 2018 on the Guardian.
`accessible', on the other hand, show an increasing trend on the Guardian.
While the data is less consistent for the Daily Express and Daily Mail, this was likely caused by a lack of data for these sources before 2007 and 2010 respectively, and a smaller overall sample size (especially for the Daily Express).

% also mention source-by-source variations
These plots also show variations in term usage between different publishers.
For example, the Guardian refers to `disability' or `disabl(ed)' by name consistently more often than the Daily Mail or Daily Express, and uses the terms `suffers(s) from' consistently less often.

%% Describe your testing strategy (unit, functional, acceptance testing; and how they are carried out). How were test cases selected?
% Not Applicable

%% Examples of specific tests and how they were carried out (e.g., using mock objects to break dependencies). Focus on the interesting cases.
% Not Applicable

%% A summary of the test results and what coverage was achieved. Detailed test reports should appear in the appendix, if they add useful information or you want to demonstrate the kinds of tests and coverage achieved.
% Not Applicable

%% If your project requires substantial evaluation of data and results, evaluation of algorithms, or other forms of testing that are not code-based, then adapt this chapter to suit.
%% This chapter will typically be 2-4 pages in length but could be more depending on the depth of testing done. If you need to do a detailed evaluation for a more mathematical or theory-based project, then this chapter could well be more substantial.


\chapter{Conclusions} \label{Conclusions}  % 2-4 pages
%% Wrap-up and final thoughts on your project. 
%% This chapter is typically 2-4 pages long but could be longer if the project work requires more extensive evaluation.

% do not write text here

\section{Achievements} \label{Achievements}
%% Summarise the achievements to confirm the project goals have been met.
%% A summary of what the project has achieved. Make sure that you address each goal set out in the Introduction chapter, to show that you have achieved what you claimed you would. Don’t leave any loose ends

The aim of this project was to explore the feasibility of exploiting NLP technologies to discover trends with regards to specific topics, with regards to the representation of disabilities and people with disabilities in British online news media.
The results to this experiment showed that this is feasible.
By analysing a dataset of 16,411 news articles related to the key terms `disabled', `disability', `handicapped', `cripple', `invalid', `accessible', `ablism', and `ableism'; the results in section \ref{Sentiment score: plots and trends} plotted trends in the variation of modelled sentiment scores across three different news publishers (Daily Express, Daily Mail, and Guardian) and over time.
The results of Mann-Whitney $U$ statistical test in section \ref{Sentiment score: statistical comparison of different sources} showed that the differences in sentiment score distributions between the three publishers are statistically significant for the `disabled' topic, and showed that `Guardian \textgreater\space Daily Mail' and `Guardian \textgreater\space Daily Express' is true for every year between 2010 and 2017.
Furthermore, the results of analysing average term occurrences of key terms, as shown in section \ref{Key terms: plots and trends}, identified increasing or decreasing trends for the terms `invalid', `handicap(ped)', and `accessibl(e)' for the Guardian; and showed variations in term usage/popularity between different publishers.

This experiment was repeated across 11 other topics (as defined in section \ref{topics}), with varying degrees of success. 
The lower sample size of news articles related to other topics in the dataset, and decreased effectiveness of the filter with regards to the selection of topics with more ambiguous keywords (e.g. `blind', `mute', `paralysed', and `stutter'), were major factors in the variability of results.
The full results of the experiment for all 12 topics are available in the Appendix.
 
The experiment also showed that it was feasible to collect and analyse a large dataset of 305,185 news articles, reduced to 48,967 articles after filtering, within a reasonable time frame using consumer-grade hardware (Intel i7-6700HQ CPU @ 2.60GHz, NVIDIA GeForce GTX 1060 GPU, 200 Mbit/s download speed).
Additionally, the vast majority of the time was spent on data collection from online sources, and an analysis of existing news articles corpora would take less time (approximately 0.37 seconds per article in extracting text features, with negligible sentiment scoring runtime using VADER).
With more powerful hardware available to institutions and large corporations, this approach should scale well to analyse corpora consisting of millions of news articles in reasonable time.

\section{Evaluation} \label{Evaluation}
%% Evaluation of the work (this may be in a separate chapter if there is substantial evaluation).
%% A critical evaluation of the results of the project (e.g., how well were the goals met, is the application fit for purpose, has good design and implementation practice been followed, was the right implementation technology chosen and so on).

This experiment showed that it was feasible to apply existing open-source NLP technologies to discover trends on a large dataset of news articles in this domain, which achieved the primary aim of this project.
The solution developed to perform this experiment delivered in performing data collection and filtering of a dataset of 305,185 news articles; and feature extraction (collecting relevant sentences and key term frequencies), sentiment scoring, statistical analysis, and data visualisation of 48,967 news articles after filtering.
The modular approach to solution design, with five separate components for each sub-task (data collection, filtering, feature extraction, sentiment scoring, and data analysis/visualisation), was ideal in this research, enabling experimentation in various components (e.g. trying multiple sentiment scorer implementations and data visualisation plots) without having to change or re-run code for other components.
This solution achieved the project's goals sufficiently as a proof of concept, and delivered meaningful results.

% Choice of technologies (what's good e.g. Python, range of libraries, plotting, etc.; what can be improved i.e. try other methods e.g. topic models for filtering, inconsistency of models used e.g. stemming with NLTK for filtering vs lemmatising with SpaCy for rule-based sentence matching)
The choice of NLP techniques and technologies, extracted features, statistical metrics, and visualisation tools are sufficient, and achieved results, but leaves room for further improvement.
The main limitation in this research was a lack of labelled data for both filtering on-topic/off-topic articles and sentiment scoring, which led to inconsistencies in both aspects.
In particular, the filter had issues with ambiguous terms such as `blind', `mute', `paralysed', and `stutter'; a supervised filter may be able to overcome this issue by taking the word's context in the sentence into account (for example, by also evaluating other words in the sentence), and also provide better evaluation of filter accuracy. 
Given this constraint, the filter worked adequately well to prepare the dataset for the experiment, and limited sampling of on-topic/off-topic articles showed that the approach was sufficiently accurate for the majority of topics.

Without labelled data for sentiment scoring, the only possible options were to use generalised supervised models trained on other domains, or to use a less complex model based on rule matching.
It was found that a less complex model based on rule matching (VADER \cite{VADER}) outperformed advanced supervised models based on statistical classifiers or neural networks trained on different domains.
However, VADER is still highly inconsistent in this domain (with an accuracy of 0.789), and although it was sufficient to discover trends in this experiment, it is likely that domain-specific supervised model would outperform it.
With a more consistent sentiment scorer, it is likely that the variance of sentiment scores within subsets would be lower, and the trend lines generated in data visualisation would be more consistent.

The usage of different NLP libraries in separate components (for example, using NLTK's stemmer in filtering and SpaCy's lemmatiser in feature extraction) also left occasional inconsistencies, such as rare cases of the feature extractor not finding any relevant sentences in articles the filter has deemed on-topic.

\section{Future Work} \label{Future Work}
%% How the project might be continued, but don't give the impression you ran out of time!
%% Future work. How could the project be developed if you had another 6 months. Take care to differentiate between what you have done to satisfy your stated project goals, and work that could be done to meet extended goals.

% Implications of NLP for future media studies, call for similar analyses to be conducted in other domains (mention additional information that can be discovered by using computational NLP and processing a much larger sample of text) 
The results showed that this computational NLP-based approach is effective in analysing news media with regards to the media representation of disabilities and people with disabilities.
This could have a substantial impact on how research will be conducted for similar studies.
Past studies (such as \cite{gold1999media, coverdale2002depictions, jones2009representations, devotta2013representations}) could be revisited using computational approaches to analyse a much larger sample size of articles, to improve the certainty and representativeness of the conclusion and possibly identify trends by varying for independent variables such as the publisher and date published.

% Purpose built sentiment model trained to a a labelled sample: in this analysis vader performed best although it's a very simple model as other, more complex pre-trained models were trained on other domains and is less generalisable. However, OpenAI's model came very close despite this, and a similar approach using deep learning or statistical ML models, trained on a labelled sample of positive/negative/neutral sentences in this domain, would have much improved sentiment score reliability (and thus quality of analyses)
As mentioned above, the lack of labelled domain-specific dataset (of sentences from news articles relating to disabilities, labelled with `positive' or `negative') for sentiment scoring limited the sentiment scores' accuracy in this experiment.
While exploring general-purpose open-source sentiment scorer implementations, it was found that a simple rule-based model outperformed the more sophisticated statistical classification or neural network models for the sentiment scoring task, as they were trained with labelled dataset from other domains.
However, one of the neural network based models (OpenAI \cite{OpenAI}) came close to the selected model (VADER \cite{VADER}), despite being trained on a completely separate domain (Amazon reviews), which suggest that a domain-specific supervised model would strongly out-perform VADER given sufficient, high-quality training data.
Thus, if an adequately large dataset of labelled sentences (likely \textgreater10,000 sentences would be required) for this domain could be compiled, future work could use this dataset to train a supervised model to improve the accuracy of the sentiment scorer.
With a more accurate and more consistent sentiment scorer, it is likely that the variance of sentiment scorers within subsets would be lower, and it would be possible to derive clearer trends and obtain statistically-significant comparisons on subsets with lower sample sizes (e.g. smaller time intervals).

Similarly, a supervised filter, trained on a labelled dataset of on-topic/off-topic news articles, would improve filtering accuracy and reduce off-topic articles in the filtered dataset, especially for trickier topics with ambiguous terms such as `blind' or `mute'.

% Expand sources - Develop API where others can use and adapt the code, change topics / keywords / key phrases / query terms, add/change sources by developing their own scrapers (with an interface for the scraper), document code and improve code readability and output file structure, output sample on-topic/off-topic articles to help them set rank threshold, maybe even an API to provide their own sentiment model(s) as parameter given an interface
Another potential application is to develop a public interface that allow users to retrieve the results of analyses performed in this experiment for other domains, given a list of topics (key terms and query terms) and/or an existing text corpora provided by the user.
The solution described in this project would have to be generalised with an interface, where users can provide their own scrapers that implement a well-documented set of functions for other news websites/sources, define their own list of topics, keywords, and query terms, and possibly even change the independent variables and subset intervals for the analysis.
It would also need to provide clear documentation on its usage and expected input structure/format.
Such a solution would provide researchers with the tools to perform similar research with ease, and possibly build on the tool as part of other projects using it as a component.

\appendix
\addtocontents{toc}{\protect\setcounter{tocdepth}{0}}  % don't show appendix sections in ToC

\printbibliography[heading=bibintoc]

\chapter{Appendix: All results}  % appendix A

\newpage
\section{Topic: `disabled'}
Key Terms: `disabled', `disability', `handicapped', `cripple', `invalid', `accessible', `ablism', `ableism', `differently abled'

\noindent Query Terms: `disabled', `disability', `ablism', `ableism', `differently abled'

\noindent Sample size, n = 16,411

\subsection{Sentiment Score Plots}
\includegraphics[width=\textwidth]{raw/disabled.png}

\subsection{Mann-Whitney $U$ Test Results ($p$-values)}
\noindent
\begin{tabu} to 1.0\textwidth { | X[c] | X[c] | X[c] | X[c] | }
	\hline
	Topic & Guardian \textgreater\space Daily Mail & Guardian \textgreater\space Daily Express & Daily Express \textgreater\space Daily Mail  \\
	\hline
	All & 2.66 * 10\textsuperscript{-167} & 2.29 * 10\textsuperscript{-35} & 0.000108  \\
	\hline
	2007 & N/A & 5.24 * 10\textsuperscript{-6} & N/A  \\
	\hline
	2008 & N/A & 0.0739 & N/A  \\
	\hline
	2009 & N/A & 2.24 * 10\textsuperscript{-5} & N/A  \\
	\hline
	2010 & 6.00 * 10\textsuperscript{-7} & 6.89 * 10\textsuperscript{-6} & 0.360 **  \\
	\hline
	2011 & 5.73 * 10\textsuperscript{-16} & 9.73 * 10\textsuperscript{-5} & 0.0355  \\
	\hline
	2012 & 4.44 * 10\textsuperscript{-17} & 0.00371 & 0.000739  \\
	\hline
	2013 & 5.03 * 10\textsuperscript{-16} & 0.00351 & 0.0827  \\
	\hline
	2014 & 1.27 * 10\textsuperscript{-9} & 0.00208 & 0.321  \\
	\hline
	2015 & 6.87 * 10\textsuperscript{-17} & 0.00855 & 0.00718  \\
	\hline
	2016 & 4.63 * 10\textsuperscript{-26} & 4.74 * 10\textsuperscript{-6} & 0.0576  \\
	\hline
	2017 & 3.39 * 10\textsuperscript{-13} & 0.00244 & 0.0867  \\
	\hline
	2018* & 0.122 & 0.120 & 0.328  \\
	\hline
\end{tabu}

\noindent * 2018 data is incomplete and would only include articles up to (approximately) end of March.

\noindent ** Indicates where the reverse assumption is true (e.g. Daily Mail \textgreater\space Daily Express instead of Daily Express \textgreater\space Daily Mail)

\subsection{Keyword Trend Plots}
\includegraphics[width=\textwidth]{raw/disabled-terms.png}

\newpage
\section{Topic: `autism'}
Key Terms: `autism', `autistic', `asperger\textbackslash's', `ASD'

\noindent Query Terms: `autism', `autistic', `asperger\textbackslash's', `ASD'

\noindent Sample size, n = 3,161

\subsection{Sentiment Score Plots}
\includegraphics[width=\textwidth]{raw/autism.png}

\subsection{Mann-Whitney $U$ Test Results ($p$-values)}
\noindent
\begin{tabu} to 1.0\textwidth { | X[c] | X[c] | X[c] | X[c] | }  
	\hline
	Topic & Guardian \textgreater\space Daily Mail & Guardian \textgreater\space Daily Express & Daily Express \textgreater\space Daily Mail  \\
	\hline
	All & 3.00 * 10\textsuperscript{-12} & 0.137 & 0.184  \\
	\hline
	2010 & 0.100 & N/A & N/A  \\
	\hline
	2011 & 0.0649 & N/A & N/A  \\
	\hline
	2012 & 0.000150 & N/A & N/A  \\
	\hline
	2013 & 1.52 * 10\textsuperscript{-5} & N/A & N/A  \\
	\hline
	2014 & 0.00162 & N/A & N/A  \\
	\hline
	2015 & 0.000508 & N/A & N/A  \\
	\hline
	2016 & 0.00760 & 0.0121 & 0.0432 **  \\
	\hline
	2017 & 0.000114 & 0.0667 & 0.352  \\
	\hline
\end{tabu}

\noindent ** Indicates where the reverse assumption is true (e.g. Daily Mail \textgreater\space Daily Express instead of Daily Express \textgreater\space Daily Mail)

\subsection{Keyword Trend Plots}
\includegraphics[width=\textwidth]{raw/autism-terms.png}

\newpage
\section{Topic: `blind'}
Key Terms: `blind', `blindness', `blindism', `visual impairment', `partially sighted', `vision loss'

\noindent Query Terms: `blind', `blindness', `visual impairment', `partially sighted', `visually impaired'

\noindent Sample size, n = 7,677

\subsection{Sentiment Score Plots}
\includegraphics[width=\textwidth]{raw/blind.png}

\subsection{Mann-Whitney $U$ Test Results ($p$-values)}
\noindent
\begin{tabu} to 1.0\textwidth { | X[c] | X[c] | X[c] | X[c] | }  
	\hline
	Topic & Guardian \textgreater\space Daily Mail & Guardian \textgreater\space Daily Express & Daily Express \textgreater\space Daily Mail  \\
	\hline
	All & 0.335 ** & 1.62 * 10\textsuperscript{-9} & 6.57 * 10\textsuperscript{-9} **  \\
	\hline
	2009 & N/A & 0.339 & N/A  \\
	\hline
	2010 & 0.388 & 0.0331 & 0.0720 **  \\
	\hline
	2011 & 0.493 & 0.0101 & 0.0149 **  \\
	\hline
	2012 & 0.00585 & 0.00586 & 0.220 **  \\
	\hline
	2013 & 0.330 ** & 0.163 & 0.158 **  \\
	\hline
	2014 & 0.0670 ** & 0.263 & 0.0857 **  \\
	\hline
	2015 & 0.181 ** & 0.214 & 0.125 **  \\
	\hline
	2016 & 0.455 & 0.00119 & 0.000966 **  \\
	\hline
	2017 & 0.421 & 3.16 * 10\textsuperscript{-5} & 1.46 * 10\textsuperscript{-5} **  \\
	\hline
	2018* & 0.0118 & 0.0919 & 0.246  \\
	\hline
\end{tabu}

\noindent * 2018 data is incomplete and would only include articles up to (approximately) end of March.

\noindent ** Indicates where the reverse assumption is true (e.g. Daily Mail \textgreater\space Daily Express instead of Daily Express \textgreater\space Daily Mail)

\subsection{Keyword Trend Plots}
\includegraphics[width=\textwidth]{raw/blind-terms.png}

\newpage
\section{Topic: `cerebral palsy'}
Key Terms: `cerebral palsy', `spastic'

\noindent Query Terms: `cerebral palsy', `spastic'

\noindent Sample size, n = 353

\subsection{Sentiment Score Plots}
\includegraphics[width=\textwidth]{raw/cerebral-palsy.png}

\subsection{Mann-Whitney $U$ Test Results ($p$-values)}
\noindent
\begin{tabu} to 1.0\textwidth { | X[c] | X[c] | X[c] | X[c] | }  
	\hline
	Topic & Guardian \textgreater\space Daily Mail & Guardian \textgreater\space Daily Express & Daily Express \textgreater\space Daily Mail  \\
	\hline
	All & 0.469 & 0.156 ** & 0.169  \\
	\hline
\end{tabu}

\noindent Insufficient sample size for year-by-year comparisons. (n=353)

\noindent ** Indicates where the reverse assumption is true (e.g. Daily Mail \textgreater\space Daily Express instead of Daily Express \textgreater\space Daily Mail)

\subsection{Keyword Trend Plots}
\includegraphics[width=\textwidth]{raw/cerebral-palsy-terms.png}

\newpage
\section{Topic: `deaf'}
Key Terms: `deaf', `deafness', `hearing impaired', `hard of hearing', `hearing loss'

\noindent Query Terms: `deaf', `deafness', `hearing impairment', `hard of hearing', `hearing impaired'

\noindent Sample size, n = 1,762

\subsection{Sentiment Score Plots}
\includegraphics[width=\textwidth]{raw/deaf.png}

\subsection{Mann-Whitney $U$ Test Results ($p$-values)}
\noindent
\begin{tabu} to 1.0\textwidth { | X[c] | X[c] | X[c] | X[c] | }  
	\hline
	Topic & Guardian \textgreater\space Daily Mail & Guardian \textgreater\space Daily Express & Daily Express \textgreater\space Daily Mail  \\
	\hline
	All & 6.30 * 10\textsuperscript{-10} & 0.141 & 0.0833  \\
	\hline
	2010 & 0.263 & N/A & N/A  \\
	\hline
	2011 & 0.00262 & N/A & N/A  \\
	\hline
	2012 & 0.00349 & N/A & N/A  \\
	\hline
	2013 & 0.0403 & N/A & N/A  \\
	\hline
	2014 & 0.000131 & N/A & N/A  \\
	\hline
	2015 & 0.00260 & N/A & N/A  \\
	\hline
	2016 & 0.0437 & N/A & N/A  \\
	\hline
	2017 & 0.0307 & 0.0480 & 0.233 **  \\
	\hline
\end{tabu}

\noindent ** Indicates where the reverse assumption is true (e.g. Daily Mail \textgreater\space Daily Express instead of Daily Express \textgreater\space Daily Mail)

\subsection{Keyword Trend Plots}
\includegraphics[width=\textwidth]{raw/deaf-terms.png}

\newpage
\section{Topic: `developmental delay'}
Key Terms: `developmental delay', `developmental disability', `developmental disorder', `learning disability', `slow learner', `intellectual disability'

\noindent Query Terms: `developmental delay', `developmental disability', `developmental disorder', `learning disability'

\subsection{Sentiment Score Plots}
\includegraphics[width=\textwidth]{raw/developmental-delay.png}

\noindent Sample size, n = 582

\subsection{Mann-Whitney $U$ Test Results ($p$-values)}
\noindent
\begin{tabu} to 1.0\textwidth { | X[c] | X[c] | X[c] | X[c] | }  
	\hline
	Topic & Guardian \textgreater\space Daily Mail & Guardian \textgreater\space Daily Express & Daily Express \textgreater\space Daily Mail  \\
	\hline
	All & 0.383 & 0.00179 & 0.00326 **  \\
	\hline
	2014 & 0.0695 ** & N/A & N/A  \\
	\hline
	2015 & 0.106 ** & N/A & N/A  \\
	\hline
	2016 & 0.000845 & N/A & N/A  \\
	\hline
	2017 & 0.251 & N/A & N/A  \\
	\hline
\end{tabu}

\noindent ** Indicates where the reverse assumption is true (e.g. Daily Mail \textgreater\space Daily Express instead of Daily Express \textgreater\space Daily Mail)

\subsection{Keyword Trend Plots}
\includegraphics[width=\textwidth]{raw/developmental-delay-terms.png}

\newpage
\section{Topic: `dyslexia'}
Key Terms: `dyslexia', `dyslexic'

\noindent Query Terms: `dyslexia', `dyslexic'

\noindent Sample size, n = 410

\subsection{Sentiment Score Plots}
\includegraphics[width=\textwidth]{raw/dyslexia.png}

\subsection{Mann-Whitney $U$ Test Results ($p$-values)}
\noindent
\begin{tabu} to 1.0\textwidth { | X[c] | X[c] | X[c] | X[c] | }  
	\hline
	Topic & Guardian \textgreater\space Daily Mail & Guardian \textgreater\space Daily Express & Daily Express \textgreater\space Daily Mail  \\
	\hline
	All & 0.0740 & 0.362 & 0.464 **  \\
	\hline
\end{tabu}

\noindent Insufficient sample size for year-by-year comparisons. (n=410).

\noindent ** Indicates where the reverse assumption is true (e.g. Daily Mail \textgreater\space Daily Express instead of Daily Express \textgreater\space Daily Mail)

\subsection{Keyword Trend Plots}
\includegraphics[width=\textwidth]{raw/dyslexia-terms.png}

\newpage
\section{Topic: `epilepsy'}
Key Terms: `epilepsy', `epileptic', `seizure'

\noindent Query Terms: `epilepsy', `epileptic'

\subsection{Sentiment Score Plots}
\includegraphics[width=\textwidth]{raw/epilepsy.png}

\noindent Sample size, n = 1,172

\subsection{Mann-Whitney $U$ Test Results ($p$-values)}
\noindent
\begin{tabu} to 1.0\textwidth { | X[c] | X[c] | X[c] | X[c] | }  
	\hline
	Topic & Guardian \textgreater\space Daily Mail & Guardian \textgreater\space Daily Express & Daily Express \textgreater\space Daily Mail  \\
	\hline
	All & 3.39 * 10\textsuperscript{-5} & 0.0579 & 0.322 **  \\
	\hline
	2014 & 0.0130 & N/A & N/A  \\
	\hline
	2015 & 0.193 & N/A & N/A  \\
	\hline
	2016 & 0.398 ** & N/A & N/A  \\
	\hline
\end{tabu}

\noindent ** Indicates where the reverse assumption is true (e.g. Daily Mail \textgreater\space Daily Express instead of Daily Express \textgreater\space Daily Mail)

\subsection{Keyword Trend Plots}
\includegraphics[width=\textwidth]{raw/epilepsy-terms.png}

\newpage
\section{Topic: `mental illness'}
Key Terms: `mental illness', `mental health', `mental disability', `mental disorder', `mental issue', `brain injured', `brain injury', `brain damaged', `psychological', `psychiatric', `emotional disorder', `behavioural disorder', `retardation', `intellectual disability', `mentally ill', `mentally disabled', `mentally handicapped'

\noindent Query Terms: `mental illness', `mental health', `mental disorder', `mental disability', `mentally ill', `mentally disabled', `mentally handicapped'

\noindent Sample size, n = 15,329

\subsection{Sentiment Score Plots}
\includegraphics[width=\textwidth]{raw/mental-illness.png}

\subsection{Mann-Whitney $U$ Test Results ($p$-values)}
\noindent
\begin{tabu} to 1.0\textwidth { | X[c] | X[c] | X[c] | X[c] | }  
	\hline
	Topic & Guardian \textgreater\space Daily Mail & Guardian \textgreater\space Daily Express & Daily Express \textgreater\space Daily Mail  \\
	\hline
	All & 3.54 * 10\textsuperscript{-76} & 1.51 * 10\textsuperscript{-9} & 0.102 **  \\
	\hline
	2010 & 0.000255 & N/A & N/A  \\
	\hline
	2011 & 8.79 * 10\textsuperscript{-11} & N/A & N/A  \\
	\hline
	2012 & 5.70 * 10\textsuperscript{-8} & 0.377 & 0.112  \\
	\hline
	2013 & 1.13 * 10\textsuperscript{-7} & 1.43 * 10\textsuperscript{-7} & 0.000132 **  \\
	\hline
	2014 & 1.19 * 10\textsuperscript{-22} & 0.108 & 0.178  \\
	\hline
	2015 & 2.77 * 10\textsuperscript{-10} & 0.0588 & 0.401 **  \\
	\hline
	2016 & 1.83 * 10\textsuperscript{-19} & 2.56 * 10\textsuperscript{-11} & 2.03 * 10\textsuperscript{-5} **  \\
	\hline
	2017 & 5.55 * 10\textsuperscript{-10} & 0.389 ** & 0.00826  \\
	\hline
	2018* & 0.0531 & 0.0413 ** & 0.00878  \\
	\hline
\end{tabu}

\noindent * 2018 data is incomplete and would only include articles up to (approximately) end of March.

\noindent ** Indicates where the reverse assumption is true (e.g. Daily Mail \textgreater\space Daily Express instead of Daily Express \textgreater\space Daily Mail)

\subsection{Keyword Trend Plots}
\includegraphics[width=\textwidth]{raw/mental-illness-terms.png}

\includegraphics[width=\textwidth]{raw/mental-illness-terms2.png}

\newpage
\section{Topic: `mute'}
Key Terms: `mute', `muteness', `mutism', `cannot speak', `difficulty speaking', `synthetic speech', `non-vocal', `non-verbal'

\noindent Query Terms: `mute', `muteness', `mutism', `non-verbal'

\noindent Sample size, n = 456

\subsection{Sentiment Score Plots}
\includegraphics[width=\textwidth]{raw/mute.png}

\subsection{Mann-Whitney $U$ Test Results ($p$-values)}
\noindent
\begin{tabu} to 1.0\textwidth { | X[c] | X[c] | X[c] | X[c] | }  
	\hline
	Topic & Guardian \textgreater\space Daily Mail & Guardian \textgreater\space Daily Express & Daily Express \textgreater\space Daily Mail  \\
	\hline
	All & 6.78 * 10\textsuperscript{-5} & 0.0311 & 0.346 **  \\
	\hline
\end{tabu}

\noindent Insufficient sample size for year-by-year comparisons. (n=456).

\noindent ** Indicates where the reverse assumption is true (e.g. Daily Mail \textgreater\space Daily Express instead of Daily Express \textgreater\space Daily Mail)

\subsection{Keyword Trend Plots}
\includegraphics[width=\textwidth]{raw/mute-terms.png}

\newpage
\section{Topic: `paralysis'}
Key Terms: `paraplegic', `quadriplegic', `spinal cord', `paraplegia', `quadriplegia', `paralysed', `paralyzed', `paralysis', `crippled', `leg braces', `wheelchair'

\noindent Query Terms: `paraplegic', `quadriplegic', `paraplegia', `quadriplegia', `paralysis'

\noindent Sample size, n = 1,461

\subsection{Sentiment Score Plots}
\includegraphics[width=\textwidth]{raw/paralysis.png}

\subsection{Mann-Whitney $U$ Test Results ($p$-values)}
\noindent
\begin{tabu} to 1.0\textwidth { | X[c] | X[c] | X[c] | X[c] | }
	\hline
	Topic & Guardian \textgreater\space Daily Mail & Guardian \textgreater\space Daily Express & Daily Express \textgreater\space Daily Mail  \\
	\hline
	All & 2.73 * 10\textsuperscript{-5} & 0.00909 & 0.0827 **  \\
	\hline
	2011 & 0.352 ** & N/A & N/A  \\
	\hline
	2012 & 0.00184 & N/A & N/A  \\
	\hline
	2013 & 0.374 & N/A & N/A  \\
	\hline
	2014 & N/A & N/A & N/A  \\
	\hline
	2015 & 0.331 & N/A & N/A  \\
	\hline
	2016 & 0.149 & N/A & N/A  \\
	\hline
	2017 & 0.00431 & N/A & N/A  \\
	\hline
\end{tabu}

\noindent ** Indicates where the reverse assumption is true (e.g. Daily Mail \textgreater\space Daily Express instead of Daily Express \textgreater\space Daily Mail)

\subsection{Keyword Trend Plots}
\includegraphics[width=\textwidth]{raw/paralysis-terms.png}

\newpage
\section{Topic: `speech impairment'}
Key Terms: `speech impairment', `stutter', `speech disability', `speech disorder', `communication disability', `difficulty speaking', `language impairment', `language disorder', `language disability', `speech impediment', `stammer'

\noindent Query Terms: `speech impairment', `stutter', `speech disorder', `speech impediment'

\noindent Sample size, n = 215

\subsection{Sentiment Score Plots}
\includegraphics[width=\textwidth]{raw/speech-impairment.png}

\subsection{Mann-Whitney $U$ Test Results ($p$-values)}
\noindent
\begin{tabu} to 1.0\textwidth { | X[c] | X[c] | X[c] | X[c] | }
	\hline
	Topic & Guardian \textgreater\space Daily Mail & Guardian \textgreater\space Daily Express & Daily Express \textgreater\space Daily Mail  \\
	\hline
	All & 0.235 & 0.0200 & 0.0546 **  \\
	\hline
\end{tabu}

\noindent Insufficient sample size for year-by-year comparisons. (n=215).

\noindent ** Indicates where the reverse assumption is true (e.g. Daily Mail \textgreater\space Daily Express instead of Daily Express \textgreater\space Daily Mail)

\subsection{Keyword Trend Plots}
\includegraphics[width=\textwidth]{raw/speech-impairment-terms.png}

% \chapter{Test Appendix B}

\end{document}